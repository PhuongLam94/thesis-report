\section{Giới thiệu}
\label{sec:gioithieu}


Chương này nhằm mục đích giới thiệu về bài toán sẽ được giải quyết trong luận văn và các khái niệm liên quan. Đầu chương sẽ nói về kỹ thuật dịch ngược, các ứng dụng của nó và những khó khăn trong quá trình dịch ngược. Phần tiếp theo trình bày bài toán đặt ra và các thách thức khi giải quyết bài toán. Phần cuối cùng tóm tắt cấu trúc của luận văn.

\subsection{Kỹ thuật dịch ngược và ứng dụng}
%Lấy từ thực tập tốt nghiệp
\subsection{Bài toán đặt ra}

Như đã đề cập ở phần trên, mục tiêu của luận văn là nghiên cứu về trình dịch ngược từ mã assembly lên mã cấp cao, các bài toán cần phải giải quyết và hiện thực giải pháp. Vì mã assembly cho kiến trúc máy khác nhau có những đặc điểm khác nhau, và đi cùng với đó là những vấn đề khác nhau cần giải quyết, nên giới hạn của luận văn sẽ là trình dịch ngược từ mã assembly 8051. Việc chọn kiến trúc máy 8051 là do 2 nguyên nhân sau:
\begin{itemize}
	\item Chip 8051 đã xuất hiện trên thị trường từ lâu, hiện tại đã không còn được sản xuất. Tuy nhiên, vẫn còn nhiều hệ thống được chạy trên đây và cần phải chuyển đổi chúng sang một kiến trúc máy khác hiện đại hơn.
	\item Chip 8051 có một số đặc điểm khác biệt so với các con chip khác trên thị trường. Vì vậy việc dịch ngược từ mã 8051 sẽ gặp nhiều khó khăn hơn, vấn đề phải giải quyết phức tạp hơn. 
\end{itemize}

Các đặc điểm khác biệt của 8051 gồm có:
\begin{itemize}
	\item Trong khi hầu hết các kiến trúc máy khác sử dụng kiểu dữ liệu byte là kiểu dữ liệu nhỏ nhất, thì 8051 cho phép lập trình viên truy xuất tới mức bit trong một số thanh ghi và kèm theo đó là các câu lệnh xử lý bit. Tuy nhiên, các thanh ghi này của 8051 cũng có thể được truy xuất ở mức byte bình thường (xem ví dụ ở đoạn mã \ref{list:list1}).
	\item Một số assembler của 8051 cho phép sử dụng tên biến. Biến này dùng để lưu các giá trị hằng số, hằng số này thường là địa chỉ một vùng nhớ kích thước 1 byte (trong luận văn này sẽ gọi tắt là biến byte) hoặc đại diện cho bit của thanh ghi (gọi tắt là biến bit). Khi lập trình, người ta thường sử dụng biến byte và biến bit này theo bộ, nghĩa là chỉ khi thanh ghi được load vào giá trị vùng nhớ quy định bởi biến byte, thì các biến bit cùng bộ mới được sử dụng (xem ví dụ ở đoạn mã \ref{list:list2}).
\end{itemize}

Từ các đặc điểm trên, ta có thể thấy bài toán lớn nhất đặt ra trong luận văn này sẽ là tìm ra được mối liên hệ giữa biến byte và biến bit trong chương trình, lấy được các bộ biến byte và biến bit đúng. Có 2 cách để biết được điều này:
\begin{itemize}
	\item Đưa ra quy định về việc khai báo biến byte và biến bit. Hiện nay, ở phần khai báo, các lập trình viên có thể khai báo các biến theo thứ tự tuỳ ý, và cũng không có quy định nào bắt buộc họ phải có phần comment chỉ rõ các biến byte và biến bit nào là cùng một bộ. Ta có thể đưa ra các mẫu khai báo cho biến byte và biến bit để trình dịch ngược có thể biết được các bộ biến bằng cách đọc theo mẫu mà không cần phân tích gì thêm. Tuy nhiên, sau khi đã xác định được các bộ biến này, cần có thêm một bước kiểm tra mã chương trình để đảm bảo rằng nguyên tắc sử dụng biến byte và biến bit được tuân thủ (Kiểm tra kiểu - Type checking). Giải pháp này có ưu điểm là đơn giản, dễ hiện thực nhưng gây bất tiện cho người dùng vì phải chuyển đổi bộ mã hiện tại về theo mẫu quy định.
	\item Dựa vào phân tích luồng dữ liệu của chương trình, tìm ra được địa chỉ vùng nhớ được load vào thanh ghi tại thời điểm sử dụng biến bit và từ đó suy ra biến byte cùng bộ với biến bit đó (Suy luận kiểu - Type inference). Với cách làm này, không cần phải thay đổi đoạn mã gốc. Tuy nhiên cách hiện thực sẽ phức tạp hơn nhiều vì có rất nhiều cách load dữ liệu vùng nhớ vào thanh ghi như: dùng trực tiếp hằng số, dùng trực tiếp biến byte, trung gian qua một thanh ghi khác... (xem ví dụ ở đoạn mã \ref{list:list3})
\end{itemize}

Cả hai giải pháp này đều được hiện thực trong từng giai đoạn của luận văn và sẽ được trình bày trong các chương tiếp theo.

\subsection{Cấu trúc luận văn}
Luận văn sẽ gồm 6 chương như sau:
\begin{itemize}
	\item Chương 1: Giới thiệu về kỹ thuật dịch ngược, bài toán đặt ra trong luận văn và cấu trúc của luận văn
	\item Chương 2: Nêu lên một số kiến thức cơ bản và các nghiên cứu liên quan đến luận văn
	\item Chương 3: Trình bày giải pháp Kiểm tra kiểu - Type checking
	\item Chương 4: Trình bày giải pháp Suy luận kiểu - Type inference
	\item Chương 5: Đánh giá kết quả của luận văn thông qua các mẫu thử (testcase)
	\item Chương 6: Kết luận
\end{itemize}