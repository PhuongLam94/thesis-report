\documentclass[12pt]{report}
\usepackage{vntex}
\usepackage{graphicx}
\usepackage{verbatim}
\usepackage[a4paper,width=150mm,top=25mm,bottom=25mm]{geometry}
\usepackage{listings}
\usepackage[
backend=bibtex,
style=numeric
]{biblatex}
\bibliography{thesis}

\title{Kỹ thuật dịch ngược}
\graphicspath{ {image/} }

\renewcommand*{\finalnamedelim}{\addcomma\addspace}
\begin{comment}
-	Lời cam đoan
-	Lời cảm ơn/ Lời ngỏ
-	Tóm tắt LV (bằng tiếng Việt và tiếng Anh)
-	Mục lục
-	Danh mục, bảng biểu, hình ảnh, ... (nếu có)
-	Nội dung LV (xem thêm phần Nội dung chi tiết ở Phần III)
-	Danh mục TL tham khảo
-	Phụ lục (nếu có)
\end{comment}
\begin{document}
	\pagenumbering{gobble}
	\chapter*{}
	\begin{center}
		ĐẠI HỌC QUỐC GIA TP. HCM\\
		TRƯỜNG ĐẠI HỌC BÁCH KHOA\\
		KHOA KHOA HỌC VÀ KỸ THUẬT MÁY TÍNH\\[10ex]
		\includegraphics[scale=0.5]{logoBK_jpg.jpg}\\[15ex]
		LUẬN VĂN TỐT NGHIỆP\\[10ex]
		\Large \textbf{KỸ THUẬT DỊCH NGƯỢC}\\[10ex]
	\end{center}
	
	
	\hspace*{7.5cm}GVHD: TS. Nguyễn Hứa Phùng\\
	\hspace*{8.15cm}GVPB: ThS. Võ Thanh Hùng\\[1ex]
	
	\hspace*{7.5cm}SVTH: Lâm Minh Phương - 51202846\\[8ex]
	
	\begin{center}
		\textit{Tp. Hồ Chí Minh, tháng 12/2016}
	\end{center}
	
	\chapter*{Lời cam đoan}
	\pagenumbering{Roman}
	
	Tôi xin cam đoan đây là công trình của tôi. Các số liệu, kết quả nêu trong báo cáo luận văn tốt nghiệp là trung thực và chưa từng được ai công bố trong bất kỳ công trình
	nào khác.\\
	Tôi xin cam đoan rằng mọi sự giúp đỡ cho việc thực hiện báo cáo luận văn tốt nghiệp này đã được cảm ơn và các thông tin trích dẫn trong báo cáo đã được ghi rõ nguồn gốc.
	
	\hspace*{8cm}\textbf{Sinh viên thực hiện}
	\\[10ex]
	\hspace*{9cm}Lâm Minh Phương
		
	\chapter*{Lời cảm ơn}
	Tôi xin gửi lời cảm ơn đến Tiến sĩ Nguyễn Hứa Phùng - Giảng viên trường Đại học Bách Khoa - Đại học Quốc Gia TP. HCM - đã giúp đỡ tôi trong suốt quá trình thực hiện luận văn này.\\
	Tôi cũng xin chân thành cảm ơn Michael James Van Emmerik, Trent Waddington và các lập trình viên đã phát triển nên nền tảng Boomerang.\\
	Ngoài ra, luận văn tốt nghiệp này là một sản phẩm kế thừa từ luận văn tốt nghiệp đề tài "Kỹ thuật dịch ngược" của Nguyễn Tiến Thành và Nguyễn Đôn Bình. Xin ghi nhận đóng góp của hai anh.\\
	Cuối cùng, xin gửi lời cám ơn đến các tác giả của tài liệu được trích dẫn trong báo cáo này. 
	
	\newpage
	\chapter*{Tóm tắt luận văn}
	Kỹ thuật dịch ngược ngày nay đã được ứng dụng nhiều trong ngành công nghiệp phần mềm. Để xây dựng một trình dịch ngược thành công, cần phải giải quyết nhiều bài toán phức tạp, chủ yếu liên quan đến việc khôi phục những thông tin không được lưu trữ ở ngôn ngữ gốc nhưng cần thiết phải có ở ngôn ngữ đích. Một trong những thông tin quan trọng nhất cần khôi phục là kiểu dữ liệu, và đã có nhiều giải pháp cho vấn đề này với độ chính xác chấp nhận được. Tuy nhiên, có những kiến trúc máy có các tính chất đặc biệt và đòi hỏi phải tạo ra một cấu trúc dữ liệu cấp cao phù hợp để đảm bảo mã cấp cao giữ nguyên được những tính chấp đó, cụ thể là cấu trúc union, và bài toán này vẫn chưa được giải quyết thấu đáo. Mục tiêu của luận văn là nghiên cứu và đưa ra những giải pháp phù hợp cho bài toán tìm kiếm union này.
	
	
	\newpage
	\tableofcontents
	
	\newpage
	\listoffigures
	
	\newpage
	\lstlistoflistings
	
	\newpage
	\pagenumbering{arabic}
	\chapter{Giới thiệu}
\label{sec:gioithieu}


Chương này nhằm mục đích giới thiệu về bài toán sẽ được giải quyết trong luận văn và các khái niệm liên quan. Đầu chương sẽ nói về kỹ thuật dịch ngược, các ứng dụng của nó và những khó khăn trong quá trình dịch ngược. Phần tiếp theo trình bày bài toán đặt ra và các thách thức khi giải quyết bài toán. Phần cuối cùng tóm tắt cấu trúc của luận văn.

\section{Kỹ thuật dịch ngược và ứng dụng}
Trong khi kỹ thuật dịch phổ biến hiện nay là dịch từ mã viết bằng ngôn ngữ cấp cao xuống mã ngôn ngữ cấp thấp hơn, kỹ thuật dịch ngược thực hiện dịch từ mã ngôn ngữ cấp thấp lên mã ngôn ngữ cấp cao hơn. Kỹ thuật dịch ngược được sử dụng rất nhiều để hỗ trợ trong quá trình phát triển và sử dụng phần mềm:

\begin{itemize}
	\item Vì một lý do nào đó, mã nguồn của một phần mềm bị mất đi. Để tiếp tục phát triển phần mềm hoặc bảo trì phần mềm đó, ta cần phải khôi phục lại mã nguồn. Nếu viết lại một chương trình mới hoàn toàn từ các tài liệu sẵn có sẽ rất mất thời gian và không đảm bảo sẽ tương đương được phần mềm cũ. Vì vậy một giải pháp phổ biến hiện nay là dựa vào file thực thi dịch ngược lại và hiệu chỉnh để có được mã nguồn mới hoàn chỉnh.
	\item Các phần mềm độc hại như virus, malware thường sẽ được giấu kín mã nguồn, nếu có được mã nguồn của chúng thì việc tìm ra phương pháp giải trừ sẽ rất dễ dàng. Ta có thể ứng dụng kỹ thuật dịch ngược để làm việc đó.
	\item Chuyển đổi chương trình chạy trên một phần cứng này sang chương trình chạy trên một phần cứng khác. Ví dụ: ta có mã đang chạy trên chip 8051, nhưng vì chip sẽ bị ngừng sản xuất trong một vài năm nữa, nên yêu cầu đề ra là tạo ra mã tương đương chạy trên một con chip khác hiện đại hơn. Để làm được điều đó, ta có thể dùng trình dịch ngược dịch mã viết cho 8051 lên một ngôn ngữ cấp cao, và sử dụng tiếp trình biên dịch để dịch mã nguồn đó thành mã của chip thay thế. Vì mã viết trong các hệ thống nhúng thường là mã assembly, nên cụ thể trình dịch ngược trong hệ thống chuyển đổi này sẽ chuyển từ mã assembly lên mã ngôn ngữ cấp cao. Và đó cũng là đối tượng nghiên cứu của luận văn này.
	
	\begin{figure}[h]
		\centering
		\includegraphics{fig12.png}
		\caption{Một ứng dụng của trình dịch ngược: chuyển đổi mã nguồn giữa các kiến trúc máy khác nhau}
	\end{figure}
	\item Phần mềm viết bằng ngôn ngữ A bắt buộc phải chuyển đổi sang ngôn ngữ B để tiếp tục bảo trì và phát triển. Ngôn ngữ A có thể là một ngôn ngữ đã ra đời từ rất lâu (ví dụ: COBOL, Basic...), hiện nay không còn người hiểu biết về ngôn ngữ đó để lập trình phần mềm. Vì vậy, cần phải chuyển đổi phần mềm sang một ngôn ngữ khác mới hơn, có nhân lực để viết tiếp (ví dụ: Java, C\#...). Quá trình này cũng được xem là dịch ngược, vì thường ngôn ngữ A ra đời trước sẽ có mức độ trừu tượng thấp hơn là các ngôn ngữ B được phát triển sau này.
\end{itemize}

Để xây dựng một công cụ dịch ngược thành công, có nhiều vấn đề cần phải giải quyết. Nhưng tựu chung lại, bài toán cơ bản nhất vẫn là khôi phục các thông tin. Khi dịch một chương trình từ mã nguồn viết bằng ngôn ngữ cấp cao xuống mã máy, có nhiều thông tin sẽ bị mất đi vì không còn cần thiết ở các ngôn ngữ cấp thấp nữa. Tuy nhiên, khi dịch lại lên ngôn ngữ cấp cao, nếu như không có những thông tin đó thì chương trình sẽ rất khó đọc, từ đó dẫn đến khó bảo trì và sửa chữa; hoặc phải chỉnh sửa chương trình đầu ra bằng tay rất nhiều. Các thông tin tiêu biểu cần khôi phục là:
\begin{itemize}
	\item Kiểu dữ liệu của biến: Đối với các chương trình viết bằng ngôn ngữ cấp cao, kiểu dữ liệu của biến có thể xem như một ràng buộc khi gán giá trị cho biến và sử dụng biến. Ví dụ khi ta khai báo một biến có kiểu dữ liệu là integer, thì ta phải gán cho biến các giá trị là số nguyên (1, 2,...) và sử dụng biến trong các phép toán có toán tử là số nguyên. Nếu ta gán cho biến một giá trị khác số nguyên (số thực, chuỗi, boolean...) hoặc sử dụng biến trong các phép toán không chấp nhận toán tử là số nguyên thì trình biên dịch sẽ phát hiện lỗi ngay ở giai đoạn đầu. Tuy nhiên, đối với mã máy, dữ liệu được lưu trong các thanh ghi hoặc vùng nhớ, nên kiểu dữ liệu không còn cần thiết và sẽ được loại bỏ trong quá trình biên dịch. Khi dịch ngược, nếu không khôi phục được kiểu dữ liệu, người lập trình sẽ rất khó khăn trong quá trình sử dụng biến vì không biết nên gán giá trị nào và sử dụng ở đâu.
	
	\item Tên của biến: Trong quá trình lập trình, tên biến có hai chức năng chính: gợi nhớ tác dụng của biến và truy xuất đến biến đó. Người lập trình khi đặt tên cho một biến thường sẽ dựa theo công dụng của biến đó để dễ dàng trong quá trình phát triển và bảo trì sau này. Một chương trình sẽ bị đánh giá là viết không tốt nếu tên biến được đặt lung tung và không có ý nghĩa nào. Ngoài ra, lập trình viên thường sẽ không quan tâm tới việc biến được lưu ở chỗ nào của vùng nhớ, khi cần truy xuất thì họ chỉ cần gọi tên biến. Ngược lại, với mã máy, để truy xuất một biến thì cần phải có tên thanh ghi hoặc địa chỉ vùng nhớ lưu biến đó. Vì vậy, tên biến đối với mã máy là không cần thiết và cũng bị loại bỏ. Tương tự như kiểu dữ liệu, nếu chương trình đầu ra không giữ được tên biến của chương trình gốc thì rất khó để phát triển và bảo trì. Giải pháp hiện nay của các trình dịch ngược là dựa vào các tài liệu sẵn có để chỉnh sửa tên biến bằng tay sau khi đã có chương trình đầu ra.
	
	\item Phân biệt giữa dữ liệu và mã điều khiển: Đặc điểm của một số mã máy (trừ mã	máy chạy trên máy ảo) là dữ liệu và các câu lệnh điều khiển có cùng một định dạng mã nhị phân và được lưu trong cùng một vùng nhớ. Vì vậy, khi dịch ngược từ mã máy lên cần phải phân biệt được phần nào của vùng nhớ là lưu các dữ liệu và phần nào là câu lệnh của chương trình.
\end{itemize}

Từ khái niệm của kỹ thuật dịch ngược, ta thấy có nhiều mức độ dịch ngược, tương ứng với những bài toán khác nhau cần giải quyết. Nếu lấy đầu ra của quá trình dịch ngược là một chương trình viết bằng ngôn ngữ cấp cao, thì đầu vào của nó có thể là: mã nhị phân, mã assembly hoặc mã của một ngôn ngữ lập trình cấp cao khác nhưng mức độ trừu tượng thấp hơn. Tùy vào mức độ trừu tượng của ngôn ngữ đầu vào, các thông tin bị mất ở mã đầu vào sẽ khác nhau. Với mã máy thì tất cả các thông tin nêu trên đều không còn. Với mã assembly, tên biến vẫn xuất hiện trong chương trình vì một số assembler cho phép có các câu lệnh khai báo biến ở mã assembly. Còn với mã ngôn ngữ cấp cao thì gần như tất cả thông tin đều có ở chương trình gốc, và vấn đề cần giải quyết là tìm ra các cấu trúc tương đương ở ngôn ngữ đích.

\section{Bài toán đặt ra}

Như đã đề cập ở phần trên, mục tiêu của luận văn là nghiên cứu về trình dịch ngược từ mã assembly lên mã cấp cao, các bài toán cần phải giải quyết và hiện thực giải pháp. Vì mã assembly cho kiến trúc máy khác nhau có những đặc điểm khác nhau, và đi cùng với đó là những vấn đề khác nhau cần giải quyết, nên giới hạn của luận văn sẽ là trình dịch ngược từ mã assembly 8051. Việc chọn kiến trúc máy 8051 là do 2 nguyên nhân sau:
\begin{itemize}
	\item Chip 8051 đã xuất hiện trên thị trường từ lâu, hiện tại sắp không còn được sản xuất. Tuy nhiên, vẫn còn nhiều hệ thống được chạy trên đây và cần phải chuyển đổi chúng sang một kiến trúc máy khác hiện đại hơn. Như vậy, nhu cầu đặt ra là có thực.
	\item Chip 8051 có một số đặc điểm khác biệt so với các con chip khác trên thị trường. Vì vậy việc dịch ngược từ mã 8051 sẽ gặp nhiều khó khăn hơn, vấn đề phải giải quyết phức tạp hơn. 
\end{itemize}

Các đặc điểm khác biệt của 8051 gồm có:
\begin{itemize}
	\item Trong khi hầu hết các kiến trúc máy khác sử dụng kiểu dữ liệu byte là kiểu dữ liệu nhỏ nhất, thì 8051 cho phép lập trình viên truy xuất tới mức bit trong một số thanh ghi và kèm theo đó là các câu lệnh xử lý bit. Tuy nhiên, các thanh ghi này của 8051 cũng có thể được truy xuất ở mức byte bình thường. Xem ví dụ ở đoạn mã \ref{list:list1}, câu lệnh số 1 gán giá trị ở vùng nhớ có địa chỉ 38H cho toàn bộ thanh ghi ACC, trong khi câu lệnh số 2 chỉ sử dụng biến số 1 của thanh ghi ACC.
	\begin{lstlisting}[caption={Một đoạn mã 8051 sử dụng cả biến bit và biến byte của thanh ghi ACC},label={list:list1}]
	MOV ACC, 38H #1
	SETB ACC.1 #2
	\end{lstlisting}
	\item Một số assembler của 8051 cho phép sử dụng tên biến. Biến này dùng để lưu các giá trị hằng số, hằng số này thường là địa chỉ một vùng nhớ kích thước 1 byte (trong luận văn này sẽ gọi tắt là biến byte) hoặc đại diện cho bit của thanh ghi (gọi tắt là biến bit). Khi lập trình, người ta thường sử dụng biến byte và biến bit này theo bộ, nghĩa là chỉ khi thanh ghi được load vào giá trị vùng nhớ quy định bởi biến byte, thì các biến bit cùng bộ mới được sử dụng (xem ví dụ ở đoạn mã \ref{list:list2}) (từ nay, khi luận văn sử dụng từ "nguyên tắc sử dụng bộ biến", nghĩa là đang đề cập đến nguyên tắc này).
		\begin{lstlisting}[caption={Một đoạn mã 8051 tuân theo nguyên tắc sử dụng bộ biến},label={list:list2}]
	#DEFINE OPTIONS #38H
	#DEFINE TESTSUP ACC.1
	public AA
	AA: 
	MOV ACC, OPTIONS
	JB TESTSUP, BB
	\end{lstlisting}
\end{itemize}

Từ các đặc điểm trên, ta có thể thấy bài toán lớn nhất đặt ra trong luận văn này sẽ là tìm ra được mối liên hệ giữa biến byte và biến bit trong chương trình, lấy được các bộ biến byte và biến bit đúng. Có 2 cách để biết được điều này:
\begin{itemize}
	\item Đưa ra quy định về việc khai báo biến byte và biến bit. Hiện nay, ở phần khai báo, các lập trình viên có thể khai báo các biến theo thứ tự tuỳ ý, và cũng không có quy định nào bắt buộc họ phải có phần comment chỉ rõ các biến byte và biến bit nào là cùng một bộ. Ta có thể đưa ra các mẫu khai báo cho biến byte và biến bit để trình dịch ngược có thể biết được các bộ biến bằng cách đọc theo mẫu mà không cần phân tích gì thêm. Tuy nhiên, sau khi đã xác định được các bộ biến này, cần có thêm một bước kiểm tra mã chương trình để đảm bảo rằng nguyên tắc sử dụng biến byte và biến bit được tuân thủ. Vì vậy, ta sẽ gọi giải pháp này là Kiểm tra kiểu - Type checking. Giải pháp này có ưu điểm là đơn giản, dễ hiện thực nhưng gây bất tiện cho người dùng vì phải chuyển đổi bộ mã hiện tại về theo mẫu quy định.
	\item Dựa vào phân tích luồng dữ liệu của chương trình, tìm ra được địa chỉ vùng nhớ được load vào thanh ghi tại thời điểm sử dụng biến bit và từ đó suy ra biến byte cùng bộ với biến bit đó. Giải pháp này được đặt tên là Suy luận kiểu - Type inference. Với cách làm này, không cần phải thay đổi đoạn mã gốc. Tuy nhiên cách hiện thực sẽ phức tạp hơn nhiều vì có rất nhiều cách load dữ liệu vùng nhớ vào thanh ghi như: dùng trực tiếp hằng số, dùng trực tiếp biến byte, trung gian qua một thanh ghi khác, dùng một biểu thức toán học có 2 vế... (xem ví dụ ở đoạn mã \ref{list:list3})
		\begin{lstlisting}[caption={Ví dụ một số mẫu câu lệnh load vùng nhớ vào thanh ghi trong 8051},label={list:list1}]
	MOV ACC, 38H
	MOV ACC, OPTIONS
	MOV ACC, @DPTR
	MOV ACC, OPTIONS+1
	\end{lstlisting}
\end{itemize}

Cả hai giải pháp này đều được hiện thực trong từng giai đoạn của luận văn và sẽ được trình bày trong các chương tiếp theo.

Ngoài ra, một công việc khác cần phải làm đó là giữ nguyên tên biến trong quá trình dịch ngược. Hiện tại trình dịch ngược Boomerang chỉ cho phép ta định nghĩa trước một số thanh ghi trong một kiến trúc máy, và đoạn mã đầu vào chỉ được sử dụng các thanh ghi đó, nếu sử dụng một cái tên nằm ngoài danh sách thanh ghi thì sẽ báo lỗi. Vì vậy, ta sẽ phải điều chỉnh cơ chế này, cho phép việc sử dụng tên biến khác và giữ nguyên chúng khi dịch ra đoạn mã ngôn ngữ cấp cao.

\section{Cấu trúc luận văn}
Luận văn sẽ gồm 6 chương như sau:
\begin{itemize}
	\item Chương 1: Giới thiệu về kỹ thuật dịch ngược, bài toán đặt ra trong luận văn và cấu trúc của luận văn.
	\item Chương 2: Nêu lên một số kiến thức cơ bản và các nghiên cứu liên quan đến luận văn. Đặc biệt, trong chương này sẽ trình bày một số kiến thức cơ bản về Boomerang, giúp người đọc dễ dàng hiểu các phần sau hơn.
	\item Chương 3: Trình bày giải pháp Kiểm tra kiểu - Type checking. Ngoài ra, trong chương này sẽ trình bày cơ chế cho phép đọc tên biến khác thanh ghi và lưu trữ chúng trong trình dịch ngược, vì đây là bước đầu tiên trong quá trình thực hiện các giải pháp.
	\item Chương 4: Trình bày giải pháp Suy luận kiểu - Type inference.
	\item Chương 5: Đánh giá kết quả của luận văn thông qua các mẫu thử (testcase).
	\item Chương 6: Kết luận.
\end{itemize}
	\chapter{Các kiến thức nền tảng và nghiên cứu liên quan}
Chương này sẽ trình bày về một số kiến thức nền tảng như: trình biên dịch, trình dịch ngược, một số kỹ thuật thường được sử dụng trong trình dịch ngược, các kỹ thuật phân tích luồng dữ liệu (data flow analysis) sẵn có được sử dụng trong luận văn, kiến trúc của trình dịch ngược Boomerang và phần mở rộng của nó. Việc vì sao lựa chọn trình dịch ngược Boomerang để hiện thực các giải pháp cho bài toán cũng sẽ được bàn đến trong chương này. \\
\section{Trình biên dịch}
Trình biên dịch (compiler) \cite{compiler} là một chương trình hoặc một bộ chương trình máy tính, có nhiệm vụ biến đổi mã nguồn được viết bằng một ngôn ngữ lập trình này (ngôn ngữ lập trình gốc) sang một ngôn ngữ lập trình khác (ngôn ngữ lập trình đích), thường sẽ có dạng nhị phân và thực thi được. \\
Các bước của trình biên dịch gồm có:
\begin{itemize}
	\item Phân tích từ vựng (lexical analysis): Đây là quá trình chuyển hóa một chuỗi các ký tự (ví dụ như các câu lệnh trong một chương trình máy tính) thành chuỗi các từ tố (token), ví dụ như: định danh, số nguyên, số thực... Giai đoạn này kiểm tra các lỗi về từ vựng, chính tả của chương trình. \\
	
	Trong hình \ref{fig:lexi}, các ký tự \textit{a}, \textit{b} được nhận diện là các định danh (identifier), \textit{10} được nhận diện là số nguyên (integer), \textit{=} và \textit{+} được nhận diện là các phép tính (operator), \textit{;} là ký tự đặc biệt kết thúc một câu lệnh.
	
	\begin{figure}[h]
		\centering
		\includegraphics{fig21.png}
		\caption{Phân tích một câu lệnh thành các token}
		\label{fig:lexi}
	\end{figure}
	
	\item Phân tích cú pháp (syntax analysis): Từ chuỗi các từ tố được tạo ra ở giai đoạn trên, một chương trình gọi là parser sẽ tạo ra một cấu trúc dữ liệu, thường là parse tree hoặc abstract syntax tree. Giai đoạn này sẽ kiểm tra các lỗi về cấu trúc ngữ pháp.
	
	Câu lệnh gán từ ví dụ trên sẽ được phân tích thành cây cấu trúc như hình \ref{fig:parser}
	
	\begin{figure}[h]
		\centering
		\includegraphics{fig22.png}
		\caption{Cây cấu trúc cho một câu lệnh gán}
		\label{fig:parser}
	\end{figure}
	
	\item Phân tích ngữ nghĩa (sematic analysis): Trong giai đoạn này, từ cây cấu trúc đã có, trình biên dịch sẽ áp dụng các luật về ngữ nghĩa để kiểm tra tính đúng đắn của chương trình. Thường sẽ là các luật về kiểu dữ liệu, kiểm tra tầm vực của biến và object binding.\\
	Tiếp tục ví dụ ở trên, trong gian đoạn này, trình biên dịch sẽ kiểm tra xem các biến \textit{a} và \textit{b} đã được khai báo chưa, tầm vực của các biến có phủ tới vị trí của câu lệnh không (ví dụ: có những biến được khai báo ở hàm \textit{A} thì sẽ không có tầm vực ở bên ngoài hàm \textit{A}), kiểu của biến có phù hợp với câu lệnh gán không (ví dụ: nếu \textit{b} có kiểu là \textbf{string} thì câu lệnh trên không hợp lệ).
	\begin{figure}[h]
		\centering
		\includegraphics{fig23.png}
		\caption{Ví dụ về lỗi kiểu biến}
		\label{fig:semerror}
	\end{figure}
	
	Trong hình \ref{fig:semerror}, cả hai đoạn mã đều hợp lệ tính đến cuối giai đoạn phân tích cú pháp. Tuy nhiên, giai đoạn phân tích ngữ nghĩa sẽ phát hiện ra đoạn mã ở bên phải không hợp lệ vì nó vi phạm các ràng buộc về kiểu.
	
	\item Tạo ra mã trung gian: Sau khi trải qua các giai đoạn phân tích và kiểm tra, trình biên dịch sẽ tiến hành sinh mã trung gian từ mã nguồn. Đặc điểm của mã trung gian là đơn giản và rất gần với mã đích, tuy nhiên con người vẫn có thể đọc và hiểu được. Việc sinh mã trung gian nhằm giảm thiểu chi phí cho trình biên dịch khi phải sinh mã đích cho nhiều kiến trúc máy khác nhau. Thay vì với mỗi kiến trúc máy, trình biên dịch phải tạo ra công cụ riêng để dịch từ mã nguồn sang mã đích, thì ở đây chỉ cần tạo ra công cụ để dịch từ mã trung gian - vốn đã rất gần với mã đích.\\
	
	
	\item Tạo mã đích: Từ mã trung gian, tùy vào kiến trúc máy sẽ thực thi chương trình, trình biên dịch sẽ tạo ra mã đích tương ứng. Giai đoạn này sẽ thực hiện các công việc như: lựa chọn câu lệnh trung gian sẽ thực hiện, quyết định các giá trị được lưu trong thanh ghi, sắp xếp thứ tự thực hiện các câu lệnh. Đầu ra của giai đoạn là mã máy có thể thực thi được.
	
	
	\item Tối ưu mã đích: Để tăng tốc độ thực hiện chương trình cũng như giảm các chi phí chạy chương trình, giai đoạn tối ưu mã đích sẽ kiểm tra và áp dụng các kỹ thuật nhằm loại bỏ mã chết, tối ưu vòng lặp, loại bỏ dư thừa... Giai đoạn này không nhất thiết chỉ thực hiện ở cuối quá trình biên dịch mà có thể nằm ở bất cứ đâu.
\end{itemize}

\section{Trình dịch ngược}
Mục tiêu của trình dịch ngược \cite{reverseengineeer} là chuyển đổi chương trình được viết bằng một ngôn ngữ cấp thấp (thường là mã máy) lên một ngôn ngữ cấp cao hơn (như C, C++...). Vì vậy, trình dịch ngược (decompiler) có thể xem như một quá trình đảo ngược của trình biên dịch (compiler). Chương trình đầu ra phải thực hiện được những chức năng tương đương như chương trình đầu vào. \\

\begin{figure}[h]
	\centering
	\includegraphics{fig24.png}
	\caption{Đoạn mã gốc và đoạn mã được dịch ngược bởi trình dịch ngược Boomerang}
\end{figure}

Qúa trình dịch ngược có thể chia thành các giai đoạn sau:
\begin{itemize}
	\item Loader: Load file cần dịch ngược, đọc từ file ra các thông tin như: loại file, loại kiến trúc máy... và xác định được ngõ vào của chương trình (tương đương với hàm main trong C).
	\item Disassembly: Mã gốc sẽ được chuyển thành mã trung gian, mã trung gian là gì thì tùy vào trình dịch ngược. Ví dụ Boomerang sẽ dùng mã trung gian là Register Transfer Language.
	\item Analysis: Sau khi đã chuyển sang mã trung gian, chương trình sẽ đi qua các bước phân tích để khôi phục lại thông tin đã mất trong quá trình biên dịch. Các phân tích thường phải có là: lan truyền biểu thức, loại bỏ mã chết, xác định nguyên mẫu hàm (function prototype), xác định kiểu dữ liệu...
	\item Code generation: Trải qua các kỹ thuật phân tích để xác định được thông tin cần thiết về dữ liệu, kiểu và luồng điều khiển chương trình, giai đoạn cuối cùng của dịch ngược là sinh ra mã chương trình bằng ngôn ngữ bậc cao. 
\end{itemize}
\begin{figure}[h]
	\centering
	\includegraphics{fig27.png}
	\caption{Các bước cơ bản của một trình dịch ngược}
\end{figure}
\section{Một số kỹ thuật tiêu biểu được sử dụng trong các công cụ dịch ngược}

\subsection{Lan truyền biểu thức}
Lan truyền biểu thức (Expression propagation) \cite{exprop} là biến đổi phổ biến nhất trong quá trình dịch ngược một đoạn code. Nguyên tắc truyền biểu thức cũng rất đơn giản: Với các câu lệnh sử dụng giá trị của một biến nào đó, ta có thể thay tên biến đó bằng biểu thức nằm bên phải câu lệnh gán biến đó.\\
\begin{figure}[h]
	\centering
	\includegraphics{fig33.png}
	\caption{Một đoạn mã trước khi thực hiện lan truyền biểu thức}
	\label{fig:33}
\end{figure}
\begin{figure}[h]
	\centering
	\includegraphics{fig34.png}
	\caption{Đoạn mã ở hình \ref{fig:33} sau khi thực hiện lan truyền biểu thức}
	\label{fig:34}
\end{figure}
Hình \ref{fig:33} và \ref{fig:34} là một ví dụ cho lan truyền biểu thức. Trong hình \ref{fig:33}, ta có các câu lệnh ở dạng mã trung gian trước khi thực hiện lan truyền biểu thức. Hình \ref{fig:34} là kết quả sau khi thực hiện lan truyền biểu thức. Ở đây ta giả sử có một biến đặc biệt là \textit{esp0} được gán giá trị là giá trị ban đầu của biến \textit{esp}. Ta sẽ thực hiện một thay thế đặc biệt ở câu lệnh số 1, thay vế phải của câu lệnh gán này - \textit{esp} - bằng biến tương đương với nó là \textit{esp0}. Sau đó, ở các câu lệnh tiếp theo, ta sẽ tiếp tục thay thế biến \textit{esp} bằng các biểu thức tương đương. Ví dụ: Ở câu lệnh số 2, biểu thức tương đương của \textit{esp} là \textit{esp0 - 4}, còn ở câu lệnh số 5, biểu thức tương đương của \textit{esp} là \textit{esp0 - 8} (do \textit{esp} đã được gán một giá trị mới ở câu lệnh số 4). Tuy nhiên, biểu thức \textit{esp0 - 8} không thể được dùng để thay thế cho biến \textit{esp} ở câu lệnh số 7 được, vì lúc đó \textit{esp} đã mang giá trị khác.\\

Như vậy, qua ví dụ trên, ta có thể thấy việc lan truyền biểu thức từ câu lệnh \textit{a} có dạng \textit{x := exp} đến một câu lệnh \textit{b} chỉ có thể được thực hiện nếu đáp ứng hai điều kiện sau:

\begin{itemize}
	\item \textit{a} phải là câu lệnh gán có vế trái là \textit{x} ở gần \textit{b} nhất. Nói cách khác, giữa \textit{a} và \textit{b} không được có bất cứ câu lệnh gán nào khác có vế trái là \textit{x}
	\item Trên tất cả các luồng đi của chương trình từ \textit{a} tới \textit{b}, không có câu lệnh gán nào có vế trái là bất kỳ biến nào được sử dụng trong \textit{a}
\end{itemize}
\begin{figure}[h]
	\centering
	\includegraphics{fig35.png}
	\caption{Đoạn mã trung gian với bốn câu lệnh gán đơn giản}
	\label{fig:35}
\end{figure}
Ở đoạn code hình \ref{fig:35}, ta có thể thực hiện lan truyền biểu thức với biến \textit{a} ở câu lệnh số 3, kết quả sẽ là \textit{c := m[b + 10] - 4;} và câu lệnh gán biến \textit{a} sẽ được loại bỏ bằng kỹ thuật loại bỏ mã chết được bàn ở phần tiếp theo. Tuy nhiên, ta không thể thực hiện lan truyền biểu thức với \textit{m[b]} ở câu lệnh số 4, vì biến \textit{c} được sử dụng trong câu lệnh gán số 2 đã được sử dụng làm vế trái trong câu lệnh gán số 3.\\

Với mã trung gian như trên, để kiểm tra hai điều kiện thỏa mãn việc lan truyền biểu thức phải mất rất nhiều thời gian, ta phải xét hết tất cả các luồng chương trình từ câu lệnh gán đến câu lệnh sử dụng biến, kiểm tra tất cả các biến được sử dụng trong câu lệnh gán. Tuy nhiên, với mã SSA - sẽ được nói đến ở mục \ref{ssa}, hai điều kiện trên sẽ được tự động thỏa mãn và không cần bất kỳ kiểm tra gì thêm.
\subsection{Loại bỏ mã chết}

Mã chết bao gồm các câu lệnh gán mà biến ở vế trái của nó không bao giờ được dùng. Cần phân biệt mã chết (dead code) với mã không bao giờ được chạy (unreachable code), là những câu lệnh mà không có bất kỳ luồng điều khiểu hợp lệ nào của chương trình đi qua (ví dụ: câu lệnh ở dưới một vòng lặp vô hạn). Việc lan truyền biểu thức sẽ dẫn đến việc có một số biến không được sử dụng, từ đó sinh ra mã chết. Kỹ thuật loại bỏ mã chết (dead code elimination) \cite{dce} sẽ bỏ đi những đoạn mã như vậy, giúp tối ưu hoá mã đầu ra của trình dịch ngược hơn. \\
Từ đoạn mã đã được lan truyền biểu thức ở hình \ref{fig:34}, ta thấy biến \textit{esp} không được sử dụng ở bất kỳ câu lệnh nào. Vì vậy, các câu lệnh gán có \textit{esp} ở vế trái sẽ được xem là mã chết và được loại bỏ. Kết quả là hình \ref{fig:36}.

\begin{figure}[h]
	\centering
	\includegraphics{fig36.png}
	\caption{Đoạn mã ở hình \ref{fig:35} sau khi loại bỏ mã chết}
	\label{fig:36}
\end{figure}

Để kiểm tra xem một biến có được sử dụng hay không, ta phải xem xét tất cả các luồng chạy hợp lệ của chương trình từ câu lệnh gán biến đến cuối chương trình, điều này phức tạp và mất nhiều thời gian. Tuy nhiên, mã SSA sẽ giúp việc kiểm tra mã chết dễ dàng hơn.
\subsection{Mã SSA} \label{ssa}

Mã SSA \cite{ssa} là một dạng mã trung gian có tính chất là: Mỗi biến hoặc vùng nhớ được định nghĩa duy nhất một lần trong toàn bộ chương trình. \\

Để chuyển từ mã RTL sang mã SSA, các biến cần phải được thay đổi tên, thường là sẽ được đánh số thứ tự đằng sau tên biến gốc. Ví dụ, nếu biến \textit{a} xuất hiện ở vế trái của 3 câu lệnh gán, thì sẽ được đánh số lần lượt là \textit{a1}, \textit{a2} và \textit{a3} như ví dụ ở hình \ref{fig:38}

\begin{figure}[h]
	\centering
	\includegraphics{fig37.png}
	\caption{Đoạn mã trung gian với 3 lần định nghĩa biến \textit{a}}
	\label{fig:37}
\end{figure}

\begin{figure}[h]
	\centering
	\includegraphics{fig38.png}
	\caption{Đoạn mã ở hình \ref{fig:38} đã được chuyển sang dạng mã SSA}
	\label{fig:38}
\end{figure}

Với tính chất của mã SSA, việc lan truyền biểu thức và loại bỏ mã chết sẽ được hiện thực rất dễ dàng.\\

Đối với lan truyền biểu thức, hai điều kiện đã được tự động thỏa mãn. Điều kiện đầu tiên thỏa mãn vì mỗi biến đều được định nghĩa duy nhất một lần, không có việc có nhiều định nghĩa cho cùng một tên biến (nếu ở mã gốc có việc đó, thì khi chuyển sang mã SSA, biến đó sẽ được đánh số để trở thành những biến khác nhau ở mỗi câu lệnh gán). Điều kiện thứ hai thỏa mãn vì chắc chắn từ câu lệnh gán một biến đến bất kỳ câu lệnh nào sử dụng biến đó, biến sẽ không được định nghĩa lại.\\

Việc loại bỏ mã chết cũng có thể thực hiện dễ dàng nhờ vào hành động thu thập thông tin về định nghĩa và sử dụng của một biến. Trong quá trình biến đổi từ mã RTL sang SSA, ta có thể xây dựng nên một bảng vị trí câu lệnh gán của một biến và vị trí các câu lệnh sử dụng biến đó. Trải qua các quá trình phân tích, nhất là lan truyền biểu thức, bảng này sẽ được cập nhật lại. Đến cuối cùng, các biến được định nghĩa nhưng không được sử dụng ở bất kỳ câu lệnh nào sẽ được xác định và loại bỏ các câu lệnh gán dư thừa đi.\\
\section{Các kỹ thuật phân tích luồng dữ liệu được sử dụng trong luận văn}
\subsection{Reaching defintions}
\label{sec:reachdef}
Reaching definitions \cite{reachingdef} là một kỹ thuật phân tích dữ liệu nhằm cho biết tại một thời điểm của chương trình, các câu lệnh gán nào còn có giá trị, hay nói cách khác là giá trị của các biến đang được gán ở những câu lệnh nào. Như ví dụ trong hình \ref{fig:reachingdefexam}, nếu áp dụng phân tích Reaching definitions, sẽ biết được tại câu lệnh số 2, có hai câu lệnh gán cho a còn có giá trị là câu lệnh số 1 và câu lệnh số 5, việc a nhận giá trị từ câu lệnh nào là tuỳ vào quá trình thực thi của chương trình.
\begin{figure}[h!]
	\centering
	\includegraphics[scale=0.75]{image/reachingDefExam}
	\caption{Một đoạn chương trình mẫu}
	\label{fig:reachingdefexam}
\end{figure}

Để tính toán tập Reaching definitions, cần đưa ra các định nghĩa sau:

\begin{itemize}
	\item Nếu một biến được gán giá trị ở câu lệnh \textit{def1}, sau đó được gán tiếp ở câu lệnh \textit{def2} sau đó, thì câu lệnh \textit{def1} đã \textbf{bị giết (killed)} bởi câu lệnh \textit{def2}.
	\item Nếu có một đường thực thi chương trình đi từ câu lệnh khai báo \textit{def1} đến một điểm \textit{p} của chương trình, mà trên đó \textit{def1} không bị giết bởi bất kỳ câu lệnh nào, thì \textit{def1} được gọi là đã \textbf{đến được (reach))} điểm \textit{p}.
\end{itemize}

Từ các định nghĩa trên, một số khái niệm mới của một khối cơ bản (B) sẽ được giới thiệu:
\begin{itemize}
	\item \textit{REACHin(B)}: Tập hợp các câu lệnh gán đến được đầu vào (entry) của B.
	\item \textit{REACHout(B)}: Tập hợp các câu lệnh gán đến được ngõ ra (exit) của B.
	\item \textit{GEN(B)}: Tập hợp các câu lệnh gán xuất hiện trong B và có thể đến được ngõ ra (exit) của B, nghĩa là vế phải trong câu lệnh đó không được gán giá trị khác ở các câu lệnh đằng sau nó.
	\item \textit{KILL(B)}: Tập hợp các câu lệnh gán mà vế phải đã được định nghĩa lại trong B.
\end{itemize}

Như vậy, có thể xem mục tiêu của Reaching definitions chính là đi tìm tập REACHin và REACHout tại một thời điểm của chương trình. Công thức để tính toán REACHin và REACHout được trình bày như sau:
\begin{equation} \label{eq:reachout}
REACHout(B) = GEN(B) \cup (REACHin(B)-KILL(B))
\end{equation}	
\begin{equation} \label{eq:reachin}
REACHin(B) = \cup_{j \in Pred(B)} REACHout(j)
\end{equation}	

Cũng giống như các phương pháp phân tích dữ liệu khác, các tập này có thể được tìm thấy thông qua một vòng lặp xét lần lượt các khối cơ bản của chương trình đến khi không còn thay đổi nào thì dừng lại. Vì phân tích Reaching definitions là một hàm đơn điệu \cite{monoreach}, nên vòng lặp này sẽ có điểm dừng. Giải thuật tính toán Reaching definitions được thể hiện ở hình \ref{fig:reachingdefalgo}

\begin{figure}[h!]
	\centering
	\includegraphics[scale=0.75]{image/reachingDefAlgo}
	\caption{Giải thuật tính Reaching definitions cho khối cơ bản}
	\label{fig:reachingdefalgo}
\end{figure}

\subsection{Lan truyền hằng số - Constant propagation}
Mục tiêu của kỹ thuật Lan truyền hằng số - Type propagation \cite{constantpropagation} là tính toán giá trị của các biến, xác định được giá trị đó có phải là một hằng số tại một thời điểm của chương trình hay không. Ví dụ như đoạn mã ban đầu \ref{list:listconstexam1}, chương trình phân tích có thể thấy giá trị của biến \textit{x} là \textbf{14}, nhưng không biết được giá trị thực sự của biến \textit{y}, cũng như biểu thức trả về là bao nhiêu. Nhờ vào việc lan truyền hằng số, các giá trị này sẽ được tính toán, như trong đoạn mã \ref{list:listconstexam2} và \ref{list:listconstexam3}.
\begin{lstlisting}[caption={Đoạn mã trước khi thực hiện lan truyền hằng số},label={list:listconstexam1}, language=c++]
int x = 14;
int y = 7 - x / 2;
return y * (28 / x + 2);
\end{lstlisting}
\begin{lstlisting}[caption={Đoạn mã sau khi thực hiện lan truyền hằng số cho biến y},label={list:listconstexam2}, language=c++]
int x = 14;
int y = 0;
return y * (28 / x + 2);
\end{lstlisting}
\begin{lstlisting}[caption={Đoạn mã sau khi thực hiện lan truyền hằng số cho biểu thức trả về},label={list:listconstexam3}, language=c++]
int x = 14;
int y = 0;
return 0;
\end{lstlisting}
Với phương pháp này, một biến có thể có ba giá trị sau:
\begin{itemize}
	\item \textit{Top}: Nghĩa là chưa biết được biến có giá trị gì.
	\item \textit{Hằng số}: Nghĩa là đã xác định được giá trị của biến là một hằng số.
	\item \textit{Bottom}: Nghĩa là biến có thể mang những giá trị khác nhau, tuỳ thuộc vào luồng chạy của chương trình.
\end{itemize}

Ở bước khai báo ban đầu của giải thuật, tất cả các biến đều được truyền vào giá trị top (chưa biết), sau đó, trải qua quá trình phân tích thì giá trị của một biến có thể được xác định là \textit{hằng số} hoặc \textit{bottom}. Trong ví dụ \ref{list:listconstexam4}, dễ dàng thấy được ở câu lệnh số 2, biến \textit{a} được gán giá trị là hằng số \textbf{4}. Và vì không có câu lệnh khai báo biến \textit{a} nào xuất hiện ở giữa, nên ở câu lệnh số 3, giá trị của biến \textit{a} cũng vẫn là hằng số \textbf{4}. Còn ở ví dụ \ref{list:listconstexam5}, giá trị của biến \textit{b} được người dùng nhập vào, nên không thể biết trước được giá trị chính xác của nó là gì. Vì vậy, cũng không thể xác định chính xác luồng đi của chương trình như thế nào. Trong quá trình thực thi, chương trình có thể đi theo nhánh \#1, cũng có thể đi theo nhánh \#2 tuỳ thuộc vào người dùng nhập gì cho biến \textit{b}. Kết quả cuối cùng là biến \textit{a} có thể mang những giá trị khác nhau ở câu lệnh dòng thứ 9. Hay nếu dựa vào định nghĩa 3 loại giá trị của biến đã nêu trên, giá trị của \textit{a} tại câu lệnh return này là \textit{bottom}. Như vậy, phân tích này cũng là một hàm đơn điệu \cite{constantpropagation} và luôn có điểm dừng.
\begin{lstlisting}[caption={Đoạn mã ví dụ biến có giá trị là hằng số},label={list:listconstexam4}, language=c++]
int a;
a = 4;
b = a*4;
\end{lstlisting}
\begin{lstlisting}[caption={Đoạn mã ví dụ biến có giá trị là bottom},label={list:listconstexam5}, language=c++]
int a;
int b;
cout<<"Enter b: ";
cin >> b;
if (b>15) #1
a = 4;
else #2
a = 5;
return a;
\end{lstlisting}

Giải thuật Constant Propagation thường dựa vào luồng đi của chương trình (Control Flow Graph) để tính toán được giá trị của các biến. Tuy nhiên, do trong các trình dịch ngược thường có một giai đoạn mã được thể hiện dưới dạng SSA, nên có thể tận dụng các mã SSA này để việc tính toán được nhanh hơn. Giải thuật này được gọi là Sparse Constant Propagation và được trình bày bên ở hình \ref{fig:constantpropagationalgo}

\begin{figure}
	\centering
	\includegraphics[scale=0.75]{image/constantPropagationAlgo}
	\caption{Giải thuật Constant propagation}
	\label{fig:constantpropagationalgo}
\end{figure}

\section{Tình hình phát triển trình dịch ngược hiện nay}
\label{sec:whyboom}
Hiện nay, có rất nhiều trình dịch ngược đã và đang được phát triển. Hầu hết đều hỗ trợ việc dịch ngược từ mã máy và có thể chia làm hai loại: trình dịch ngược nhận đầu vào là mã máy chạy trên máy ảo (ví dụ: mã máy được dịch từ các chương trình viết bằng Java, C\#) và trình dịch ngược nhận đầu vào là mã máy chạy trên máy thật. \cite{decompilersi} Loại thứ nhất có số lượng nhiều hơn, lý do có thể vì mã máy ảo còn lưu giữ được khá nhiều thông tin từ chương trình gốc, điển hình như tên biến toàn cục (xem hình \ref{fig:ilspy}). Vì vậy, bài toán cần giải quyết để xây dựng một trình dịch ngược dạng này không nhiều. Ngược lại, mã máy thật đã bị mất hầu hết các thông tin từ chương trình gốc, nên việc khôi phục thông tin ở các trình dịch ngược từ mã máy thật khá phức tạp. Trong hình \ref{fig:boomerang}, tên biến ở chương trình gốc đã bị mất đi. Ngoài ra, cấu trúc vòng lặp cũng thay đổi từ \textit{for} sang \textit{while}. Đó là do ở mã máy, cấu trúc vòng lặp ở ngôn ngữ cấp cao đã được dịch thành các câu lệnh kiểm tra điều kiện và jump, nên Boomerang phải sử dụng các thuật toán phân tích luồng điều khiển để tạo ra cấu trúc vòng lặp mới, đôi khi có thể không trùng khớp với cấu trúc vòng lặp ban đầu.

\begin{figure}[h]
	\centering
	\includegraphics{fig25.png}
	\caption{Một đoạn mã được dịch ngược bởi trình dịch ngược ILSpy. Tên biến static \textit{abc} được giữ nguyên như mã gốc}
	\label{fig:ilspy}
\end{figure}
\begin{figure}[h]
	\centering
	\includegraphics{fig26.png}
	\caption{Một đoạn mã được dịch bởi Boomerang}
	\label{fig:boomerang}
\end{figure}

Một số trình dịch ngược phổ biến có thể kể đến là:

\begin{itemize}
	\item dcc \cite{dcc}: Là một trình dịch ngược từ mã máy, đây được xem là một trong những trình dịch ngược đầu tiên và vẫn còn được phát triển tới bây giờ.
	\item ILSpy \cite{ilspy}: Là một trình dịch ngược cho .NET, input là các file assembly được dịch từ chương trình .NET, được phát triển bởi isharpcode. Hiện nay ILSpy vẫn đang được tiếp tục phát triển và thêm các tính năng mới.
	\item Procyon \cite{procyon}: Là một trình dịch ngược cho Java. Trước đây lựa chọn hàng đầu để dịch ngược mã Java là JAD (Java decompiler), tuy nhiên hiện nay JAD đã ngừng phát triển và mã nguồn không còn mở nữa. Một số trình dịch ngược khác được phát triển và Procyon là một đại diện tiêu biểu.
	\item Boomerang \cite{boomeranghomepage}: Là trình dịch ngược từ mã máy, mục tiêu là tạo ra một trình dịch ngược không quan tâm tới ngôn ngữ viết ra chương trình gốc. Boomerang đã ngừng phát triển từ năm 2006 do hai lập trình viên chính bắt đầu làm việc cho một công ty mà lĩnh vực nghiên cứu của họ trùng lặp với Boomerang.
\end{itemize}

Trong số các trình dịch ngược nêu trên, cần tìm ra trình dịch ngược phù hợp nhất để làm nền tảng cho việc hiện thực những giải pháp mà luận văn đề ra. Để tìm ra trình dịch ngược đó, phải có sự phân tích, đánh giá sự phù hợp của những trình dịch ngược thông qua một số tiêu chí như sau:
\begin{itemize}
	\item Phù hợp với bài toán cần giải quyết: Do bài toán cần giải quyết là tìm kiếm kiểu union ở các đoạn mã assembly, nên những trình dịch ngược đã có sẵn cơ chế chấp nhận mã assembly sẽ được tính điểm. (Tiêu chí 1)
	\item Là general decompiler: Như đã trình bày ở trên, có hai loại decompiler là loại dành cho mã máy ảo và loại dành cho mã máy thật. Vì loại dành cho mã máy thật sẽ có tính ứng dụng rộng rãi hơn, nên sẽ được đánh giá cao hơn. (Tiêu chí 2)
	\item Mã máy lưu trữ được nhiều thông tin gốc của chương trình: Đối với các mã máy ảo, một số thông tin của chương trình gốc như tên biến, kiểu dữ liệu có thể được lưu trữ. Điều này giúp cho việc khôi phục thông tin sẽ dễ dàng hơn. (Tiêu chí 3)
	\item Có người hỗ trợ: Nếu như trình dịch ngược này đã có người nghiên cứu trước và có thể hướng dẫn trực tiếp thì việc thực hiện luận văn sẽ dễ dàng hơn. (Tiêu chí 4)
	\item Có tài liệu đầy đủ: Tương tự, việc có tài liệu đầy đủ cũng sẽ giúp quá trình hiện thực giải thuật trên nền tảng này diễn ra nhanh chóng hơn. (Tiêu chí 5)
	\item Viết bằng ngôn ngữ quen thuộc: Nếu trình dịch ngược được viết trên một ngôn ngữ quen thuộc với tác giả của luận văn, thì việc lập trình chỉnh sửa trình dịch ngược sẽ thuận tiện hơn là phải làm quen với một ngôn ngữ mới. (Tiêu chí 6)
\end{itemize}
Phần đánh giá này được thể hiện ở bảng 2.1
\begin{table}[h!]
	\centering
	\begin{tabular}{ |p{0.7cm}| p{2cm}| p{1.5cm}| p{1.5cm}| p{2cm}| p{1.5cm}| p{1.5cm}
			| p{1.5cm}| p{1.5cm}| p{1.5cm}| }
	\hline
		
		STT & Tên trình dịch ngược & Tiêu chí 1 (1đ) & Tiêu chí 2 (1đ) & Tiêu chí 3 (2đ) & Tiêu chí 4 (5đ) & Tiêu chí 5 (3đ) & Tiêu chí 6 (2đ) & Tổng điểm\\
		\hline
		1 & Boomerang & 1 & 1 &0 & 5 & 2 & 0 & 9\\
	\hline
		2 & ILSpy & 0 & 0 & 2 & 0& 3 & 2 & 7\\
		\hline
		3 & dcc & 1 &0&0&0&0&0&1\\
		\hline
		4&Procyon&0&0&2&0&3&2&7\\
		\hline
	\end{tabular}


\label{table:abc}
	\caption{Bảng đánh giá các trình dịch ngược}
\end{table}


\section{Trình dịch ngược Boomerang}
\subsection{Kiến trúc của Boomerang}
Phần này sẽ giới thiệu về cấu trúc code của Boomerang \cite{boomeranghomepage}, giúp ích cho việc trình bày các giải pháp của bài toán ở chương kế.

Về mặt tổng thể, Boomerang gồm có các phần sau:
\begin{figure}[h]
	\centering
	\includegraphics[scale=1.3]{fig32.png}
	\caption{Cấu trúc các khối lớn của Boomerang}
	\label{fig:boomstruct}
\end{figure}

Khi đọc vào một chương trình assembly, Boomerang sẽ lưu trữ chúng dưới cấu trúc được mô tả trong hình \ref{fig:assemblyinfo}.

\begin{figure}[h!]
	\centering
	\includegraphics[scale=0.6]{assemblyInfo}
	\caption{Cấu trúc dữ liệu lưu trữ mã assembly trong Boomerang}
	\label{fig:assemblyinfo}
\end{figure}

Lưu ý, trong AssemblyArgument, giá trị thực sự của tham số được lưu vào một union có tên là Arg. Giá trị này có thể là một chuỗi (đối với trường hợp thanh ghi hoặc tên biến), một số nguyên, một số thực hoặc một structure đại diện cho bit (bao gồm tên thanh ghi và vị trí của bit). Đoạn mã \ref{list:assemblyinfo} thể hiện điều đó.
\begin{lstlisting}[caption={Đoạn mã mô tả cách biểu diễn giá trị của tham số trong Boomerang}, label={list:assemblyinfo}, language=c++]
struct bits{
	char* reg;
	int pos;
}
union Arg{
	int i;
	float f;
	char* c;
	bits bit;
}
\end{lstlisting}

Một lưu ý là cấu trúc nói trên chỉ áp dụng với các chương trình viết bằng ngôn ngữ assembly, còn khi viết bằng mã máy thì Boomerang sẽ sử dụng một cấu trúc khác, tuy nhiên, khi chuyển đổi chương trình đầu vào thành mã trung gian thì chỉ có duy nhất một loại cấu trúc cho tất cả các loại mã máy. Prog là tương ứng với toàn bộ chương trình. Một Proc là một hàm, BasicBlock đại diện cho một khối cơ bản mà ở đó không có một câu lệnh rẽ nhánh nào (ví dụ như if, hoặc vòng lặp...). Statement là một câu lệnh và Expr là các biểu thức trong chương trình. Ngoài ra còn có các class đại diện cho kiểu dữ liệu. Việc thực hiện các phân tích chủ yếu diễn ra tại Proc, vì vậy, các thay đổi trong luận văn này cũng chủ yếu được thực hiện bằng các hàm của Proc. Cấu trúc mã trung gian của Boomerang được thể hiện ở hình \ref{fig:progboomerang}\\
\begin{figure}
	\centering
	\includegraphics[scale=0.6]{progBoomerang}
	\caption{Cách lưu trữ một chương trình dưới dạng mã trung gian của Boomerang}
	\label{fig:progboomerang}
\end{figure}


\subsection{Phần mở rộng của Boomerang}

Boomerang có nhiều ưu điểm, tuy nhiên chỉ hỗ trợ dịch ngược từ mã máy. Như đã nói ở chương đầu, việc dịch ngược từ mã asembly lên mã cấp cao là một nhu cầu có thật trong thực tế. Để giải quyết điều đó, một phần mở rộng Boomerang \cite{lvtn} đã được phát triển để nó chấp nhập đầu vào là các file assembly và đi qua một công cụ là Boomerang Toolkit để biến mã assembly thành mã trung gian theo chuẩn của Boomerang. Tiếp sau đó, mã trung gian này được đưa vào phần backend của Boomerang và tiếp tục các quá trình phân tích, sinh mã. Ưu điểm của phương pháp này là ta chỉ cần can thiệp vào phần front end của Boomerang để tạo ra được mã trung gian và tận dụng được các ưu điểm của nền tảng này về thuật toán tốt ở phần back end.\\

\begin{figure}[h]
	\centering
	\includegraphics[scale=0.8]{fig51.png}
	\caption{Minh họa phần mở rộng Boomerang. Ngoài file nhị phân, Boomerang đã có thể nhận đầu vào là file assembly}
\end{figure}

Boomerang đã được mở rộng để nhận được file đầu vào là assembly, tuy nhiên, phần mở rộng này còn một số vấn đề chưa được giải quyết như: chưa giữ được tên biến đầu vào, chưa có cách sinh mã phù hợp với kiểu union. Vì vậy, để hiện thực luận văn, cần phải chỉnh sửa thêm các điểm nêu trên trong Boomerang. Việc chỉnh sửa này sẽ được trình bày ở phần \ref{sec:boomchange}.

	\chapter{Kiểm tra kiểu - Type checking}
\label{chap:typechecking}
Chương này trình bày cách giải quyết bài toán Kiểm tra kiểu thông qua các bước sau:
\begin{itemize}
\item Mở rộng ngôn ngữ assembly để lưu trữ thông tin về kiểu union.
\item Kiểm tra tính hợp lệ của việc sử dụng kiểu union trong thân chương trình.
\end{itemize}

Các phần tiếp theo của chương sẽ trình bày lần lượt các bước này.

\begin{comment}
\section{Chỉnh sửa để Boomerang chấp nhận việc sử dụng biến không phải thanh ghi}
\subsection{Cơ chế lưu trữ tên thanh ghi hiện nay của Boomerang}
Khi chuyển đổi từ mã assembly sang mã trung gian, Boomerang sẽ dùng một class con của Expr để biểu diễn thanh ghi. Cụ thể là class Location, và gọi phương thức static của class Location là Location::regOf(int num). Ta sẽ truyền vào phương thức này một con số đại diện cho thanh ghi đó. Cặp số - tên thanh ghi này được lưu vào một từ điển, để sau này khi thực hiện phân tích xong thì sẽ chuyển lại từ thanh ghi thành biến cục bộ.

Trong phần giải mã từ mã assembly sang mã trung gian, có một hàm để map giữa tên thanh ghi và con số đại diện cho nó, đó là hàm map\_sfr(string name).
\begin{lstlisting}[caption={Một số phần mã trong hàm map\_sfr},label={list:listmapsfr}]
if (name == "R0") return 0;
else if (name == "R1") return 1;
else if (name == "R2") return 2;
...
else return -1;

\end{lstlisting}

Sau khi trải qua các quá trình phân tích và đến giai đoạn in ra mã đầu ra, trình dịch ngược sẽ gọi hàm getRegName trong class FrontEnd để trả lại tên ban đầu của thanh ghi. Trong hàm getRegName sẽ lấy từ điển tên thanh ghi - số đại diện được quy định sẵn của mỗi phần giải mã cho các kiến trúc máy khác nhau, tìm tên thanh ghi tương ứng với con số đó và trả về.
\begin{lstlisting}[caption={Phần mã trong hàm getRegName},label={list:listgetregname}]
std::map<std::string, int, std::less<std::string> >::iterator it;
for (it = decoder->getRTLDict().RegMap.begin();	 it != decoder->getRTLDict().RegMap.end(); it++)
if ((*it).second == idx) 
return (*it).first.c_str();
return NULL;a
\end{lstlisting}


Như vậy, có thể thấy với các tên biến không được quy định trước, hàm map\_sfr sẽ trả về giá trị -1, và vì giá trị -1 sẽ không có trong từ điển của phần giải mã, nên hàm getRegName sẽ trả về NULL, dẫn đến trình dịch ngược sẽ bị lỗi runtime và dừng ngay lập tức.\\

%lấy mã Boomerang ban đầu về, hiện kết quả khi sử dụng biến đầu vào

Vì số lượng tên biến là rất nhiều, nên ta không thể sử dụng phương pháp thêm mới các tên biến vào từ điển được quy định sẵn được, mà phải có cách để trình dịch ngược linh động hơn, chấp nhận bất kỳ các tên nào được sử dụng trong mã assembly. Giải pháp đưa ra là ngoài việc sử dụng từ điển thanh ghi được quy định sẵn, ta sẽ lập thêm một bảng tên biến, thành phần bao gồm các cặp tên biến - số đại diện. Trong giai đoạn giải mã, khi hàm map\_sfr được gọi, nếu tên truyền vào nằm trong các thanh ghi đã quy định sẵn, thay vì trả về giá trị -1 thì ta sẽ tạo ra một giá trị random và đưa chúng vào bảng tên biến ở trên. Ngoài ra, còn có một đoạn mã kiểm tra biến được sử dụng đã được khai báo bằng câu lệnh \#DEFINE chưa (ngoại trừ một số biến đặc biệt được tự sinh). \\
\begin{lstlisting}[caption={Phần mã mới được bổ sung trong hàm map\_sfr},label={list:listmapsfrnew},language=c++]
bool isDefined = false;
map<char*, AssemblyArgument*>::iterator it;
for (it = replacement.begin(); it!=replacement.end(); it++){
if(strcmp((*it).first, name.c_str()) == 0 ){
isDefined = true;
break;
}
}
if (isDefined || name.find("specbits") != string::npos ){
if (symbolTable->find(name) == symbolTable->end()){
bool existed = false;
int num;
do{
num = std::rand()%200+31;
map<string, int>::iterator it;
for (it = symbolTable->begin(); it!=symbolTable->end(); it++){
bool cond1 = (*it).second == num;
bool cond2 = (byteVar != -1 && byteVar>=num);
bool cond3 = (bit != -1 && bit>=num);
if (cond1 || cond2 || cond3){
existed = true;
continue;
} else {
existed = false;
}
}
} while (existed); 
(*symbolTable)[name] = num;
if (name.find("specbits") != string::npos){
std::cout<<"Name: "<<name<<", "<<num<<endl;
}
return num;
} else {
return symbolTable->find(name)->second;
}
}
else {
std::cout<<"ERROR: "<<name<<" HAS NOT BEEN DEFINED YET"<<endl;
exit(1);
}
\end{lstlisting}
Tương ứng với sự thay đổi ở hàm map\_sfr,  ở hàm getRegName, ngoài việc dò trong từ điển quy định trước, ta sẽ thêm một đoạn mã để dò trong bảng tên biến. 
\begin{lstlisting}[caption={Phần mã mới được bổ sung trong hàm getRegName},label={list:listgetregnamenew},language=c++]
std::map<string,int>::iterator symIt;
for (symIt = decoder->getSymbolTable().begin(); symIt != decoder->getSymbolTable().end(); symIt++){
	if ((*symIt).second == idx){
		return (*symIt).first.c_str();
	}
}
\end{lstlisting}
Như vậy, vấn đề giữ nguyên tên biến được giải quyết mà không ảnh hưởng nhiều tới trình dịch ngược.
%đoạn mã đầu vào assembly và mã đầu ra giữ nguyên được tên biến
\end{comment}
\section{Mở rộng ngôn ngữ assembly}
Để kiểm tra được tính hợp lệ khi sử dụng kiểu union trong đoạn mã đầu vào, trình dịch ngược cần phải được cung cấp thông tin về các kiểu union thông qua một phương thức nào đó. Có hai hướng để giải quyết vấn đề này là:
\begin{itemize}
	\item Đưa ra một cấu trúc khai báo mới cho ngôn ngữ assembly. Cấu trúc khai báo này có thể tương tự như khai báo union ở ngôn ngữ cấp cao. Tuy nhiên, cách làm này sẽ làm cho đoạn mã assembly không thể compile được vì các assembler không chấp nhận cấu trúc mới thêm vào đó.
	\item Cho phép người lập trình đưa các thông tin về kiểu union vào trong phần chú thích theo một mẫu quy định từ trước. Với giải pháp này, đoạn mã không bị ảnh hưởng vì chú thích là một thành phần đã có sẵn trong ngôn ngữ assembly, và khi compile thì các assembler sẽ bỏ qua phần chú thích.
\end{itemize}
Như vậy, giải pháp thứ hai là tối ưu hơn và sẽ được áp dụng trong luận văn này. Mẫu chú thích được viết cho 8051 được thể hiện ở đoạn mã \ref{list:listdeclarevar}
\begin{lstlisting}[caption={Mẫu khai báo bộ biến},label={list:listdeclarevar}]
;BEGIN DEFINE
;DEFINE BYTE
[byte var declare]
;DEFINE BITS
[eight bit var declares]
;END DEFINE
\end{lstlisting}
Tuy nhiên, cũng giống như assembler, các trình dịch ngược hầu như sẽ bỏ qua phần chú thích khi đọc đoạn mã đầu vào, và để cho trình dịch ngược có thể rút trích được thông tin từ phần chú thích thì phải thực hiện một số chỉnh sửa trong giai đoạn lexer và parser, cụ thể là:
\begin{itemize}
	\item Chỉnh sửa lexer để nhận biết các token là những chú thích đặc biệt. Như trong ví dụ \ref{list:listdeclarevar}, các chú thích \textbf{";BEGIN DEFINE"}, \textbf{";DEFINE BYTE"}, \textbf{";DEFINE BITS"} không thể được đọc vào như là token \textit{COMMENT} bình thường, mà phải có những token riêng biệt cho chúng để phục vụ giai đoạn parser. Đoạn mã \ref{list:8051lexer} được viết để chạy trên thư viện \textit{flex++} thể hiện điều đó. Có thể thấy, khi bắt được một chú thích, thay vì trả về token \textit{COMMENT} như trước đó, phần lexer mới này sẽ kiểm tra nội dung chú thích, nếu trùng với các chú thích đặc biệt thì sẽ trả về token tương ứng. 
\end{itemize}
\begin{changemargin}{0cm}{0cm} 
	\begin{lstlisting}[caption={Phần lexer được chỉnh sửa để nhận biết các chú thích đặc biệt},label={list:8051lexer}]
	(\;.*)  { 
		string beginDefine = ";BEGIN DEFINE";
		string endDefine = ";END DEFINE";
		string defineByte = ";DEFINE BYTE";
		string defineBits = ";DEFINE BITS";
		string val = strdup(yytext);
		if (beginDefine == val)
			return BEGINDEFINE;
		else if (endDefine == val)
			return ENDDEFINE;
		else if (defineByte == val)
			return DEFINEBYTE;
		else if (defineBits == val)
			return DEFINEBITS;
		else return COMMENT;
		}
	\end{lstlisting}

\end{changemargin} 
\begin{itemize}
	\item Chỉnh sửa parser để nhận vào cấu trúc khai báo union. Tiếp tục với ví dụ mẫu khai báo \ref{list:listdeclarevar}, nếu phần parser chỉ đọc các câu lệnh khai báo biến bình thường, thì không thể xác định được các union mà người dùng khai báo trước. Như vậy, cần chỉnh sửa parser để bắt được những mẫu khai báo đặc biệt. Phần parser viết bằng thư viện \textit{bison++} cho mẫu khai báo \ref{list:listdeclarevar} được trình bày ở đoạn mã \ref{list:8051parser}. Trước khi chỉnh sửa, phần parser này chỉ có luật \textit{define}, đọc vào từng câu lệnh khai báo độc lập. Sau khi được chỉnh sửa lại, luật \textit{definebit} có độ ưu tiên cao hơn sẽ đọc vào cấu trúc khai báo union gồm nhiều câu lệnh khai báo khác nhau.
\end{itemize}
\begin{changemargin}{0cm}{0cm} 
	\begin{lstlisting}[caption={Đoạn mã parser nhận biết các mẫu khai báo union},label={list:8051parser}]
	definebit: BEGINDEFINE END_LINE DEFINEBYTE END_LINE 
					define DEFINEBITS END_LINE defineeachbit{8} 
					ENDDEFINE END_LINE;
	defineeachbit: DEFINE ID bit END_LINE;
	define:	DEFINE ID expressions END_LINE;
	\end{lstlisting}
\end{changemargin} 
\begin{itemize}
	\item Chỉnh sửa các hành động sau khi parser nhận biết được những cấu trúc khai báo union. Sau khi parser đã bắt được các mẫu khai báo union, cần lập trình để lưu trữ thông tin về các union đó vào dữ liệu của chương trình. Cấu trúc được dùng để lưu trữ thông tin về union này là \textit{UnionDefine}, được trình bày ở đoạn mã \ref{list:listuniondefine}, trong đó, \textit{byteVar} để lưu trữ tên của union đó, còn \textit{bitVar} chứa các thành phần bit thuộc union và số thứ tự của bit mà thành phần đó truy xuất. Đoạn mã \ref{list:8051parser2} thể hiện phần code đưa các khai báo union từ mã đầu vào thành các thực thể \textit{UnionDefine} tương ứng.
\end{itemize}
\begin{changemargin}{0cm}{0cm} 
	\begin{lstlisting}[caption={Cấu trúc dữ liệu để lưu trữ một union},label={list:listuniondefine},language=c++]
	class UnionDefine{
		public:
			char* byteVar;
			map<int, char*>* bitVar;
	};
	\end{lstlisting}
	\begin{lstlisting}[caption={Đoạn mã parser bao gồm các hành động sau khi nhận biết được union},label={list:8051parser2}]
	definebit: BEGINDEFINE END_LINE DEFINEBYTE END_LINE 
					define DEFINEBITS END_LINE defineeachbit{8} 
					ENDDEFINE END_LINE
	{
		UnionDefine* ut = new UnionDefine();
		$5 -> expList -> pop_back();
		ut->byteVar = $5 -> expList -> back() -> 
		argList.back()->value.c;
		ut->bitVar = bitVar;
		unionDefine1 -> push_back(ut);
		bitVar = new map<char*, int>();
	};
	defineeachbit: DEFINE ID bit END_LINE {
		std::string temp($3->value.c);
		char c =  temp.at(temp.size()-1);
		int num = c - '0';
		(*bitVar)[$2] = num;
	};
	define: DEFINE ID expressions END_LINE 
	{ 
		AssemblyLine* line = new AssemblyLine();
		line -> expList = new list<AssemblyExpression*>();
		line->kind = INSTRUCTION;
		line->name = "DEFINE";
		AssemblyExpression* expr1 = new AssemblyExpression();
		expr1 -> kind = UNARY;
		Arg a;
		a.c=$2;
		expr1 -> argList.push_back(new AssemblyArgument(6, a));
		line -> expList->push_back(expr1);
		line->expList ->push_back(expr);//*/
		expr = new AssemblyExpression();
		$$=line;
	};
	\end{lstlisting}
\end{changemargin} 
	
Như vậy, sau khi thực hiện các chỉnh sửa trên, trình dịch ngược đã có thể đọc vào các thông tin về union được người dùng cung cấp ở phần chú thích của đoạn mã đầu vào, đồng thời lưu trữ thông tin đó vào cấu trúc dữ liệu \textit{UnionDefine}. Tuy đã xác định được những union nào được khai báo trong mã đầu vào, nhưng cần phải có một bước để kiểm tra những union đó có được sử dụng hợp lý hay không.
\section{Kiểm tra tính hợp lệ của việc sử dụng kiểu union}
Như đã trình bày ở phần \ref{sec:challenge}, ngôn ngữ assembly không cho phép thao tác trực tiếp trên các vùng nhớ, mà phải thông qua một thanh ghi trung gian và việc sử dụng một thành phần \textit{y} thuộc union \textit{x} chỉ hợp lệ khi mà thanh ghi trung gian đang mang kiểu union \textit{x} đó. Thông thường, mỗi kiến trúc máy sẽ có một thanh ghi trung gian đặc trưng, ví dụ như với 8051 là thanh ghi \textit{ACC}. Như vậy, quá trình kiểm tra kiểu union sẽ gồm 2 bước:
\begin{itemize}
	\item Xác định kiểu của thanh ghi trung gian tại mỗi điểm của chương trình.
	\item Kiểm tra việc truy xuất thành phần của union có hợp lý hay không.
\end{itemize}
\subsection{Xác định kiểu của thanh ghi trung gian}
Trong quá trình nghiên cứu, có 2 giải pháp được đưa ra để xác định kiểu của thanh ghi:
\begin{itemize}
	\item Sử dụng phương pháp phân tích Reaching definitions (đã được giới thiệu ở phần \ref{sec:reachdef}), lập trình thêm phần mở rộng để xác định thông tin về kiểu từ tập câu lệnh gán tìm được.
	\item Sử dụng phương pháp Lan truyền kiểu - Type propagation để lan truyền kiểu của thanh ghi theo luồng đi của chương trình.
\end{itemize}
Hai giải pháp này được trình bày lần lượt ở các phần tiếp theo.
\subsubsection{Phân tích Reaching definitions kết hợp phần mở rộng}
Phân tích Reaching definitions cho biết được tại một thời điểm của chương trình, các câu lệnh gán nào đang còn có giá trị, nhưng thông tin đó là chưa đủ để biết được kiểu dữ liệu của thanh ghi trung gian. Nếu câu lệnh tìm thấy được là một câu lệnh gán trực tiếp vùng nhớ có địa chỉ là một biến khai báo trước cho thanh ghi, thì dễ dàng xác định được kiểu của thanh ghi là union mang tên biến đó. Tuy nhiên, thực tế cho thấy biểu thức quy định địa chỉ vùng nhớ cho thanh ghi có nhiều dạng khác nhau, cụ thể là:
\begin{itemize}
		\item Một hằng số.
	\item Một thanh ghi, giá trị của thanh ghi có thể được khai báo ở các câu lệnh trước đó.
	\item Một biểu thức có hai vế, các vế của biểu thức có thể là một biến, một thanh ghi hoặc một hằng số.
\end{itemize}
Ví dụ về các câu lệnh gán vùng nhớ cho thanh ghi viết bằng mã 8051 được thể hiện ở đoạn mã \ref{list:listhardtestcase}. Phương pháp Reaching definitions sẽ không thể xử lý được khi gặp các câu lệnh gán này. Ở câu lệnh số 1, phân tích có thể lấy được giá trị \textbf{38H}, nhưng không thể xác định được biến nào đã được khai báo giá trị \textbf{38H}. Ở câu lệnh số 2, Reaching definitions không thể biết được giá trị của thanh ghi \textit{DPTR} là bao nhiêu. Ở câu lệnh số 3, việc xử lý lại càng phức tạp hơn, vì nếu chỉ đơn giản lấy biểu thức vế phải \textit{OPTIONS+1} của khai báo ra, không thể nào biết được giá trị thực sự của nó là bao nhiêu. \\
\begin{lstlisting}[caption={Một số câu lệnh gán trên 8051 mà Reaching definitions không xử lý được},label={list:listhardcase}]
MOV A, 38H ;1
MOV A, @DPTR ;2
MOV A, OPTIONS+1 ;3
\end{lstlisting}
Ngoài ra, trong đoạn mã \ref{list:ifcond}, tại thời điểm câu lệnh sử dụng biến bit \textit{TESTSUPS}, có hai câu lệnh khai báo biến \textit{a}. Đối với phương pháp Reaching definitions, nó sẽ xem như có hai giá trị mà biến \textit{a} có thể mang. Tuy nhiên, nếu xét kỹ hơn, thì sẽ thấy là cả hai giá trị đó đều là biến \textit{OPTIONS}, và thực chất tại thời điểm này \textit{a} chỉ mang một giá trị, cho dù luồng đi của chương trình có như thế nào. Như vậy, phương pháp Reaching definitions sẽ bỏ qua những trường hợp như thế này và dẫn đến việc độ chính xác sẽ không được cao.
\begin{lstlisting}[caption={Đoạn mã có nhiều câu lệnh khai báo cho ACC đến được một điểm của chương trình nhưng tất cả đều cùng giá trị},label={list:ifcond}]
if (...){
a = *(OPTIONS);
...
} else {
a = *(OPTIONS);
...
}
TESTSUPS = 1;
\end{lstlisting}
Để khắc phục những khuyết điểm trên của phương pháp Reaching definitions, cần có một phần mở rộng để xử lý được những trường hợp gán phức tạp. Ý tưởng cho phần mở rộng đó như sau:
\begin{itemize}
	\item Đối với địa chỉ vùng nhớ là hằng số: Vì trong quá trình parser, trình dịch ngược có lưu trữ thông tin về tên biến và giá trị của biến đó vào một danh sách dữ liệu, nên có thể tìm kiếm trong danh sách đó ra được tên biến và đó cũng là tên của union mà thanh ghi đang mang.
	\item Đối với địa chỉ vùng nhớ là một thanh ghi khác: Kiểm tra tập các câu lệnh gán còn giá trị tại thời điểm đó của chương trình, sẽ tìm ra được biểu thức giá trị của thanh ghi đó. Áp dụng giải thuật của phần mở rộng lên biểu thức đó để tìm ra giá trị thực sự của thanh ghi và trả về. Nếu giá trị trả về là một biến được khai báo trước, thì tên biến là kiểu union cần tìm.
	\item Đối với địa chỉ vùng nhớ là một biểu thức hai toán hạng: Hiện tại, phần mở rộng không thể xử lý được trường hợp này, do việc tính toán giá trị của một biểu thức hai toán hạng là khá phức tạp. Đây là một trong những khuyết điểm của phần mở rộng.
\end{itemize}
Cụ thể các bước giải thuật của phần mở rộng được trình bày ở hình \ref{fig:reachdefextendalgo}.
\begin{figure}
	\centering
	\includegraphics[width=\linewidth]{image/reachdefextendalgo}
	\caption{Giải thuật cho hàm findRegValue - phần mở rộng của phân tích Reaching definitions}
	\label{fig:reachdefextendalgo}
\end{figure}

Theo giải thuật nêu trên, cần truyền vào cho hàm \textit{findRegValue} các giá trị sau đây:
\begin{itemize}
	\item \textit{regName}: Tên thanh ghi cần tìm kiểu dữ liệu
	\item \textit{isMain}: Biến bool cho biết đây có phải là thanh ghi cần xác định kiểu không hay chỉ là thanh ghi trung gian của quá trình xác định kiểu đó. Nếu là thanh ghi cần xác định kiểu, thì biểu thức được xét đến là địa chỉ của vùng nhớ được gán cho thanh ghi, còn nếu chỉ là thanh ghi trung gian, thì biểu thức được xét đến là giá trị trực tiếp của thanh ghi.
	\item \textit{REACHin}: Tập các câu lệnh gán có giá trị đến thời điểm đó của chương trình. Tập này được xây dựng nhờ vào giải thuật Reaching definitions.
	\item \textit{listDefine}: Đây là danh sách lưu trữ các biến được khai báo ở đầu chương trình. Danh sách này gồm các cặp tên biến - giá trị biến.
\end{itemize}

Như vậy, sau khi trải qua phần mở rộng này, nếu thanh ghi trung gian đang mang một kiểu union được khai báo trước thì hàm \textit{findRegValue} sẽ trả về \textbf{tên của kiểu union} đó. Còn nếu không phải thì giá trị trả về là \textbf{NULL}. Đến đây, việc xác định kiểu union của thanh ghi trung gian đã giải quyết xong.\\

Tuy đã xử lý được hầu hết các trường hợp của phép gán vùng nhớ cho thanh ghi, nhưng việc có thêm một phần mở rộng này sẽ làm cho tốc độ xử lý trình dịch ngược giảm đi. Cộng thêm việc bản thân giải thuật Reaching definitions đã có độ phức tạp cao, tổng thời gian xử lý cho bước này của trình dịch ngược là khá lớn. Vì vậy, nhu cầu đặt ra là có một phương pháp khác có độ chính xác tương đương Reaching definitions cộng phần mở rộng, nhưng có thời gian xử lý thấp hơn. Phương pháp này sẽ được trình bày tiếp theo đây.

\subsubsection{Phân tích Lan truyền kiểu - Type propagation}
Phương pháp phân tích Lan truyền kiểu - Type propagation có thể xem là một đóng góp mới của luận văn tốt nghiệp này. Phương pháp này được đặt ra nhằm giải quyết hạn chế về mặt thời gian xử lý của Reaching definitions kết hợp phần mở rộng. Có hai nguyên nhân chính làm thời gian xử lý của phương pháp trên bị chậm là:
\begin{itemize}
	\item Sau khi đã chạy vòng lặp quét các câu lệnh của chương trình, vẫn chưa thể kết luận được kiểu dữ liệu mà thanh ghi đang mang. Phải trải qua một giai đoạn xử lý dữ liệu nữa thì mới rút trích ra được thông tin đó.
	\item Với cách chạy vòng lặp cổ điển của các phương pháp phân tích dữ liệu, nếu ở một câu lệnh nào đó có sử thay đổi về tập \textit{REACHin} và \textit{REACHout}, thì toàn bộ các câu lệnh khác trong chương trình đều phải tính toán lại 2 tập trên dù chúng có bị ảnh hưởng hay không.
\end{itemize}
Như vậy, để tăng nhanh thời gian xử lý, phương pháp Lan truyền kiểu có ý tưởng như sau:
\begin{itemize}
	\item Xem thông tin về kiểu union của thanh ghi là một kiểu dữ liệu để lan truyền qua các câu lệnh của chương trình. Như vậy, kết thúc quá trình chạy vòng lặp, đã có thể xác định được thanh ghi trung gian đang mang kiểu union gì mà không cần phải xử lý thêm.
	\item Thay vì sử dụng phương pháp chạy vòng lặp, bất cứ câu lệnh nào có thay đổi gì thì sẽ tính toán lại toàn bộ chương trình, thì phương pháp sử dụng worklist sẽ được thay thế. Với phương pháp này, chỉ các câu lệnh có trong worklist mới được tính toán, và khi có thay đổi ở một câu lệnh nào đó, thì chỉ những câu lệnh bị ảnh hưởng bởi sự thay đổi này mới được đưa vào worklist và tính toán tính.
	\item Sử dụng mã SSA nhằm giúp cho việc tính toán thuận tiện hơn. Chi tiết về công dụng của mã SSA sẽ được phân tích rõ hơn sau khi thuật toán chi tiết được trình bày. Trong các trình dịch ngược, luôn có một giai đoạn mã đầu vào được thể hiện ở dạng SSA để phục vụ cho các quá trình phân tích dữ liệu, nên sẽ không mất thời gian chuyển đổi từ mã thường sang mã SSA.
\end{itemize}
Các ý tưởng trên được thể hiện cụ thể qua giải thuật ở hình \ref{fig:typepropagationalgo}.\\
\begin{figure}[h!]
	\centering
	\includegraphics[width=\linewidth]{image/typePropagationAlgo}
	\caption{Giải thuật của phân tích Type propagation}
	\label{fig:typepropagationalgo}
\end{figure}
Từ giải thuật trên, có thể thấy mã SSA mang đến những lợi ích sau:
\begin{itemize}
	\item Tiết kiệm không gian lưu trữ. Vì mỗi biến SSA chỉ được định nghĩa một lần duy nhất, nên chỉ cần một bảng lưu trữ tên biến - kiểu union cho toàn bộ chương trình, không cần thiết ở mỗi câu lệnh phải lưu một tập vào và một tập ra như phương pháp Reaching definitions.
	\item Giúp cho việc tìm các câu lệnh bị ảnh hưởng bởi sự thay đổi của của một câu lệnh nào đó dễ dàng hơn. Ở bước cuối của vòng lặp, phải tìm các câu lệnh có sử dụng biến ở vế trái của câu lệnh thay đổi. Nếu đoạn mã đang ở dạng thông thường, thì việc xác định tập câu lệnh này rất khó khăn, do một biến có thể được định nghĩa lại nhiều lần nên phải kiểm tra câu lệnh nào đang sử dụng biến được định nghĩa tại câu lệnh được ghi nhận thay đổi. Nhưng với mã SSA, mỗi biến chỉ được gán một lần duy nhất trong toàn bộ chương trình, nên việc tìm kiếm sẽ dễ dàng hơn rất nhiều.
\end{itemize}

Như vậy, phương pháp Lan truyền kiểu này sẽ làm giảm đáng kể thời gian xử lý của chương trình so với phương pháp Reaching definitions. Tuy nhiên, khuyết điểm của Lan truyền kiểu là vẫn chưa xử lý được trường hợp biểu thức vế phải có hai toán hạng. Để giải quyết được vấn đề này, cần có một phân tích khác mạnh hơn và phân tích đó sẽ được giới thiệu ở chương tiếp theo.
\subsection{Kiểm tra việc truy xuất thành phần của union có hợp lý hay không}

\label{sec:laststep}
Sau khi đã xác định được kiểu của thanh ghi trung gian ở từng thời điểm của chương trình, bước tiếp theo sẽ là kiểm tra tại thời điểm truy xuất thành phần của một union, thanh ghi trung gian có đang mang đúng kiểu union đó không. Việc kiểm tra này là khá đơn giản do đã có thông tin:
\begin{itemize}
	\item Kiểu union và các thành phần của nó từ phần khai báo
	\item Kiểu union của thanh ghi trung gian thông qua quá trình phân tích
\end{itemize}
Như vậy, chỉ cần dò tìm và so trùng hai dữ liệu trên để đưa ra kết quả cuối cùng. Các bước kiểm tra này trình bày ở hình \ref{fig:checkunionsteps}.
\begin{figure}
	\centering
	\includegraphics[width=\linewidth]{image/checkUnionSteps}
	\caption{Quá trình kiểm tra một câu lệnh sử dụng bit}
	\label{fig:checkunionsteps}
\end{figure}
Một ví dụ về việc truy xuất thành phần union không hợp lý được thể hiện ở đoạn mã \ref{list:invalid8051}. Trong đoạn mã này, phần khai báo cho thấy \textit{TESTSUPS} là một thành phần thuộc union \textit{OPTIONS}, tuy nhiên ở câu lệnh số 2, khi thực hiện truy xuất thành phần \textit{TESTSUPS}, thanh ghi trung gian là \textit{ACC} lại mang một kiểu union khác là \textit{OPTIONS2}. Như vậy, việc truy xuất này là không hợp lý và phải được cảnh báo để người dùng sửa chữa lại đoạn mã chương trình.
	\begin{lstlisting}[caption={Đoạn mã 8051 chứa một truy xuất thành phần union không hợp lý},label={list:invalid8051}]
;BEGIN DEFINE
;DEFINE BYTE
#DEFINE OPTIONS #38
;DEFINE BITS
#DEFINE TESTSUPS ACC.1
...
;END DEFINE
...
MOV A, OPTIONS2 ;1
SETB TESTSUPS ;2
\end{lstlisting}

Như vậy, với bài toán Kiểm tra kiểu, thông tin về kiểu union đã có sẵn ngay từ đầu và chỉ cần kiểm tra xem người lập trình có thực hiện việc truy xuất các thành phần của union hợp lý không trước khi sinh ra mã ở ngôn ngữ cấp cao. Giải pháp cho bài toán yêu cầu can thiệp vào trình dịch ngược ít và hiện thực dễ dàng. Tuy nhiên, giải pháp còn nhiều hạn chế như phương pháp phân tích dữ liệu chưa đạt độ chính xác cao, cần người dùng phải chỉnh sửa lại chú thích theo mẫu quy định...




	\chapter{Suy luận kiểu - Type inference}
Sau khi đã hiện thực giải pháp Kiểm tra kiểu, nhận thấy rằng giải pháp này còn nhiều hạn chế và có thể cải tiến thêm, luận văn đã phát triển thêm một giải pháp mới tốt hơn, không bắt buộc người dùng phải thay đổi code của mình theo mẫu khai báo, đó là Suy luận kiểu. Với giải pháp này, bằng các phép phân tích dữ liệu, trình dịch ngược sẽ tự động tìm ra được các bộ biến được sử dụng trong chương trình. Ngoài bước chung của hai giải pháp là chỉnh sử trình dịch ngược để giữ nguyên tên biến đã được trình bày ở giải pháp trước, các bước của giải pháp Suy luận kiểu gồm có:
\begin{enumerate}
	\item Dùng một phương pháp phân tích luồng dữ liệu để biết được giá trị của thanh ghi ACC tại mỗi điểm của chương trình.
	\item Đi qua các câu lệnh sử dụng biến bit, ghi nhận giá trị hiện tại của thanh ghi ACc tại câu lệnh đó và đưa biến bit đó vào bộ biến phù hợp.
	\item Thêm vào các union tương ứng với bộ biến tìm ra được, thay thế các thanh ghi đại diện cho biến bit bằng truy xuất đến union tương ứng, cũng như thay thế vị trí sử dụng thanh ghi ACC.
\end{enumerate}
Như vậy, giải pháp này đã bỏ qua được bước đầu tiên, quy định mẫu khai báo, của giải pháp Suy luận kiểu. Ngoài ra, kỹ thuật phân tích Reaching definitions như đã trình bày ở chương trước có một số khuyết điểm, vì vậy, ở giải pháp này, chúng ta sẽ phân tích và tìm ra một kỹ thuật khác toàn diện hơn. Điều này sẽ được trình bày ở phần đầu tiên của chương, phần tiếp theo sẽ nói về cách quét các câu lệnh sử dụng biến bit và đưa thông tin vào một cấu trúc dữ liệu phù hợp.

\section{Phân tích Constant propagation}
Ở chương trước, phương phán phân tích Reaching definitions đã được đề cập đến, tuy nhiên, phương pháp này chỉ áp dụng được cho trường hợp gán một biến byte trực tiếp cho thanh ghi ACC, vì vậy ta cần tìm một phương pháp khác phù hợp hơn. Phương pháp đạt yêu cầu cần phải tính toán được chính xác giá trị hiện có của thanh ghi ACC cho dù biểu thức bên phải của phép gán là gì. Ngoài ra, nếu thanh ghi ACC có thể có mang những giá trị khác nhau ở một câu lệnh sử dụng bit, thì mặc nhiên nguyên tắc bị vi phạm. Như vậy, ta chỉ xét tới các trường hợp giá trị ở thanh ghi ACC là một giá trị cố định, có thể tính toán được trước khi thực thi chương trình. Từ các yêu cầu trên, ta kết luận được phương pháp phân tích phù hợp nhất trong trường hợp này là Lan truyền hằng số - Constant propagation. Phương pháp này cho phép tính toán giá trị của các biến, cho biết được gía trị đó có phải là một hằng số tại một thời điểm của chương trình hay không. Ví dụ như đoạn mã ban đầu \ref{list:listconstexam1}, có thể rõ ràng thấy giá trị của biến x là 14, nhưng ta không biết được giá trị thực sự của biến y, cũng như biểu thức trả về là bao nhiêu. Nhờ vào việc lan truyền hằng số, các giá trị này sẽ được tính toán, như trong đoạn mã \ref{list:listconstexam2} và \ref{list:listconstexam3}.
\begin{lstlisting}[caption={Đoạn mã trước khi thực hiện lan truyền hằng số},label={list:listconstexam1}, language=c++]
 int x = 14;
int y = 7 - x / 2;
return y * (28 / x + 2);
\end{lstlisting}
\begin{lstlisting}[caption={Đoạn mã sau khi thực hiện lan truyền hằng số cho biến y},label={list:listconstexam2}, language=c++]
int x = 14;
int y = 0;
return y * (28 / x + 2);
\end{lstlisting}
\begin{lstlisting}[caption={Đoạn mã sau khi thực hiện lan truyền hằng số cho biểu thức trả về},label={list:listconstexam3}, language=c++]
int x = 14;
int y = 0;
return 0;
\end{lstlisting}
Với phương pháp này, một biến có thể có ba giá trị sau:
\begin{itemize}
	\item Top: Nghĩa là chưa biết được biến có giá trị gì.
	\item Hằng số: Nghĩa là đã xác định được giá trị của biến là một hằng số.
	\item Bottom: Nghĩa là biến có thể mang những giá trị khác nhau, tuỳ thuộc vào luồng chạy của chương trình.
\end{itemize}

Ở bước khai báo ban đầu của giải thuật, tất cả các biến đều được truyền vào giá trị top (chưa biết), sau đó, trải qua quá trình phân tích thì giá trị của một biến có thể được xác định là hằng số (như giá trị của biến a tại câu lệnh số 3, đoạn mã \ref{list:listconstexam4}) hoặc là bottom (như giá trị biến a tại câu lệnh số 9, đoạn mã \ref{list:listconstexam5}).
\begin{lstlisting}[caption={Đoạn mã ví dụ biến có giá trị là hằng số},label={list:listconstexam4}, language=c++]
int a;
a = 4;
b = a*4;
\end{lstlisting}
\begin{lstlisting}[caption={Đoạn mã ví dụ biến có giá trị là bottom},label={list:listconstexam5}, language=c++]
int a;
int b;
cout<<"Enter b: ";
cin >> b;
if (b>15)
	a = 4;
else
	a = 5;
return a;
\end{lstlisting}

Như vậy, khi áp dụng vào trình dịch ngược Boomerang, mục tiêu của giải thuật này là để tìm ra được ở mỗi điểm của chương trình, giá trị thật sự của địa chỉ vùng nhớ mà thanh ghi ACC đang mang là gì.

Có nhiều cách thực hiện Constant propagation, vì trong Boomerang, có một giai đoạn code trung gian được giữ ở dạng SSA, nên ta sẽ chọn cách phân tích Sparse constant propagation để giảm thiểu thời gian xử lý. Và việc phân tích này sẽ được thực hiện ở cuối giai đoạn SSA, khi các phân tích khác đã hoàn tất. Giải thuật của phân tích Sparse constant propagation gồm có các bước được trình bày ở hình \ref{fig:constantpropagationalgo}

\begin{figure}
	\centering
	\includegraphics[scale=0.75]{image/constantPropagationAlgo}
	\caption{Giải thuật Constant propagation đã được điều chỉnh phù hợp với yêu cầu của trình dịch ngược}
	\label{fig:constantpropagationalgo}
\end{figure}

Để thể hiện giá trị của một biến có thể thuộc ba loại là top, hằng số hoặc bottom, ta sẽ tạo ra một class mới trong Boomerang, đó là ConstantVariable. Đoạn mã của class này được trình bày bên dưới.
\begin{lstlisting}[caption={Đoạn mã thể hiện class ConstantVariable},label={list:listconstexam5}, language=c++]
class ConstantVariable{
	public:
	int type; //1: top, 2: constant, 3: bottom
	Exp* variable;
	ConstantVariable(){
		type = 3;
	}
};
\end{lstlisting}
Như vậy, ta có thể thấy kết quả của giải thuật này là tạo ra được một sơ đồ map giữa một biến SSA và một thực thể ConstantVariable thể hiện giá trị của biến đó. \\

Ngoài ra, ta sẽ áp dụng code pattern Visitor để tính toán được giá trị của các biểu thức nằm ở vế phải của lệnh gán. Visitor đó được đặt tên là EvalExpressionVisitor. Vì các biểu thức có thể viết ở mức assembly khá đơn giản, nên ta chỉ cần viết hàm visit cho các loại biểu thức sau:

\begin{itemize}
	\item Const: Là biểu thức hằng số. Hàm visit này chỉ đơn giản trả về giá trị hằng số nếu đây là một hằng số nguyên.
	\item Binary: Là biểu thức có 2 vế. Hàm visit sẽ visit từng vế của biểu thức, và nếu cả hai vế đều là hằng số, thì sẽ thực hiện phép tính cộng trừ nhân chia hai hằng số đó để ra được kết quả cuối cùng.
	\item RefExp: Loại biểu thức này chứa một biểu thức khác, kèm theo câu lệnh khai báo biểu thức đó. Cụ thể, ta chỉ xét loại RefExp có biểu thức con là một biến. Ta sẽ tìm trong sơ đồ hiện tại để lấy ra giá trị của biến đó. Nếu không có trong sơ đồ, ta sẽ tìm trong bảng lưu trữ dữ liệu các câu lệnh \#DEFINE để xem đó có phải là một biến đã được khai báo trước trong chương trình đầu vào không.
	\item TypedExp: Loại biểu thức để ép kiểu một biểu thức nào đó thành kiểu mong muốn. Với trường hợp này, ta sẽ visit biểu thức con và trả về giá trị của biểu thức đó.
\end{itemize}


Như vậy, với phương pháp phân tích này, ta có thể giải quyết được vấn đề vế phải của phép gán thanh ghi ACC không chỉ đơn giản là một biến byte. Ngoài ra, nó còn nhận biết được các biểu thức có giá trị giống nhau mặc dù hình thức bên ngoài khác nhau. Xem ví dụ các câu lệnh ở đoạn mã \ref{list:listdiffassignacc}. Câu lệnh gán số 1 và số 2 thực chất đều gán cho ACC giá trị vùng nhớ có địa chỉ quy định bởi biến OPTIONS. Nếu thực hiện phân tích Reaching definitions ở giải pháp trước, trình dịch ngược sẽ không thể biết được điều này. Tuy nhiên, ở giai đoạn này, vì trình dịch ngược sẽ tính toán được ở cả hai câu lệnh, ACC đều mang giá trị của vùng nhớ có địa chỉ là 38H. Và ở những bước tiếp theo, trình dịch ngược sẽ đối chiếu giá trị 38H với bảng lưu trữ dữ liệu và biết được biến OPTIONS đại diện cho giá trị đó.

\begin{lstlisting}[caption={Một số câu lệnh gán cho ACC có giá trị vế phải bằng nhau},label={list:listdiffassignacc}]
#DEFINE OPTIONS #38H
...
MOV ACC, OPTIONS
MOV ACC, 38H
\end{lstlisting}

Một lưu ý là trong trường hợp này, ta không xét đến giá trị của thanh ghi ACC, mà ta chỉ xét đến giá trị của địa chỉ vùng nhớ thanh ghi ACC đang lưu giữ. Như vậy, chỉ có các câu lệnh dạng MOV ACC, [biểu thức] sẽ được xét đến. Khi gặp câu lệnh gán có dạng MOV ACC, \#[biểu thức] thì đoạn mã phân tích sẽ xem như giá trị của ACC không phải là hằng số (bottom). Như vậy, với các biến khác, giá trị lưu trong thực thể ConstantExpression tương ứng với biến đó là giá trị thực sự của biến, còn riêng với thanh ghi ACC, giá trị đó được hiểu là giá trị địa chỉ vùng nhớ mà thanh ghi ACC được load vào.
\section{Quét các câu lệnh sử dụng biến bit}

Sau khi đã biết được giá trị địa chỉ vùng nhớ mà thanh ghi ACC đang nắm giữ, ta sẽ chuyển sang bước quét các câu lệnh sử dụng biến bit. Quá trình thực hiện bước này là khá giống nhau giữa hai giải pháp. Tuy nhiên, điểm khác biệt là nếu ở giải pháp Kiểm tra kiểu, do thực hiện phân tích Reaching definitions, nên ta sẽ biết chính xác được là tại câu lệnh sử dụng biến bit, thanh ghi ACC giữ giá trị của vùng nhớ có địa chỉ quy định bởi biến byte nào, còn với giải pháp Suy luận kiểu, do sử dụng phân tích Constant propagation, ta chỉ biết chính xác giá trị địa chỉ vùng nhớ thanh ghi ACC đang được load vào, mà không biết biến byte nào đại diện cho giá trị đó. Để biết được cụ thể biến byte nào tương ứng, ta sẽ phải dò trên bảng lưu trữ dữ liệu từ các câu lệnh \#DEFINE, và nếu mỗi lần gặp một câu lệnh sử dụng bit đều làm vậy thì tốc độ sẽ không cao.\\

Giải pháp để nâng cao tốc độ xử lý là trong quá trình quét câu lệnh sử dụng bit, ta sẽ không tìm kiếm mối quan hệ giữa biến bit - biến byte, mà chỉ tìm kiếm mối quan hệ giữa biến bit - một giá trị trực tiếp nào đó. Sau khi tìm ra được tất cả các bộ này trong chương trình đầu vào, ta sẽ lần lượt tìm kiếm các biến byte tương ứng với giá trị trực tiếp và thay thế vào. Để thực hiện được điều đó, ta vẫn sẽ sử dụng cấu trúc lưu trữ UnionDefine, nhưng ngoài thành phần byteVar, ta sẽ thêm vào một thành phần mới là byteVarValue. Thành phần này sẽ lưu trữ giá trị trực tiếp ứng với các biến bit lưu ở bitVar. Ở phần cuối của giai đoạn phân tích, ta sẽ tìm ra được biến byte tương ứng và đưa vào trường byteVar.\\

\begin{lstlisting}[caption={Đoạn mã mới của class UnionDefine},label={list:listnewuniondefine},language=c++]
	class UnionDefine{
	public:
	char* byteVar;
	map<int, char*>* bitVar;
	int byteVarValue;
	void prints(){
	cout << "Byte var: " << byteVar <<endl;
	cout << "Bit vars: "<<endl;
	map<int, char*>::iterator mi;
	for (mi = bitVar->begin(); mi != bitVar -> end(); mi++){
	cout << (*mi).second << ": " << (*mi).first << endl;
	} } };
\end{lstlisting}

Quá trình quét câu lệnh và thu thập dữ liệu của giải pháp Suy luận kiểu cũng tương tự như giải pháp Kiểm tra kiểu, nhưng thay vì ghi nhận biến bit và biến byte rồi kiểm tra chúng có cùng một bộ như khai báo hay không, ta sẽ kiểm tra trước hai biến đó có vi phạm nguyên tắc sử dụng hay không (ví dụ như biến bit đó đã thuộc một bộ khác trước đó, hoặc ở bit vị trí đó của bộ biến của biến byte đã ghi nhận một biến bit khác...), nếu không thì ta sẽ ghi nhận mối quan hệ này vào danh sách UnionDefine đang lưu trữ.

\begin{figure}[h]
\centering
\includegraphics[width=0.7\linewidth]{image/stepUnionMaking}
\caption{Các bước kiểm tra và ghi nhận dữ liệu vào danh sách UnionDefine}
\label{fig:stepunionmaking}
\end{figure}


Sau khi đã quét hết các câu lệnh ở các procedure, trước khi chuyển đổi các UnionDefine thành các khai báo ở ngôn ngữ C và thêm vào danh sách các biến toàn cục của program như giải pháp trước, ta phải thêm vào một bước chuyển đổi từ giá trị thành biến byte đại diện cho giá trị đó. Điều này có thể được thực hiện bằng cách chạy vòng lặp qua bảng lưu trữ các câu lệnh \#DEFINE đã được thiết lập từ quá trình parse mã đầu vào. Nếu như có một giá trị nào đó chưa được khai báo ở câu lệnh \#DEFINE, ta sẽ đặt tên mới cho biến byte đó là LOCATION\_[giá trị]. Ví dụ như LOCATION\_56

Bước cuối cùng của giải pháp này là thay thế các biến bit thành truy xuất tới union tương ứng, loại bỏ câu lệnh gán thanh ghi ACC và thay thế các vị trí sử dụng thanh ghi ACC thành các biến byte tương ứng. Cách thực hiện tương tự như trong giải pháp Suy luận kiểu đã trình bày ở phần \ref{sec:laststep} chương \ref{chap:typechecking}

Như vậy, giải pháp này đã giải quyết phần lớn các vấn đề đặt ra của bài toán. 
	\chapter{Kiểm thử}

Bất kỳ một sản phẩm nào đều cần phải được kiểm tra trước khi công bố, luận văn này cũng không phải là một ngoại lệ. Để đảm bảo chất lượng được đánh giá một các khách quan nhất, một hệ thống testcase với các loại tình huống được phân bổ một cách khoa học nhất sẽ được đưa ra, sau đó cho chạy thử qua cả 2 giải pháp của luận văn. Phần một của chương này sẽ trình bày về các testcase đó, phần hai sẽ trình bày cách kiểm tra chúng trên 2 giải pháp, và phần cuối sẽ là kết quả kiểm tra.

\section{Hệ thống testcase}
Có các tiêu chí phân loại testcase như sau:
\begin{itemize}
	\item Loại biểu thức được gán vào thanh ghi ACC (tiêu chí I)
	\item Cách truy xuất bit của thanh ghi (tiêu chí II)
	\item Có vi phạm nguyên tắc sử dụng bộ biến hay không (tiêu chí III)
\end{itemize}
Với mỗi tiêu chí, ta sẽ có các trường hợp sau đây:

Tiêu chí I:
\begin{enumerate}
	\item Một giá trị trực tiếp
	\item Giá trị ở một vùng nhớ có địa chỉ là một biến byte
	\item Giá trị ở một vùng nhớ có địa chỉ là một giá trị trực tiếp
	\item Giá trị ở một vùng nhớ có địa chỉ là một thanh ghi
	\item Giá trị ở một vùng nhớ có địa chỉ là một biểu thức 2 vế. Mỗi vế có thể là một biến byte, một thanh ghi, hoặc một giá trị trực tiếp
\end{enumerate}

Tiêu chí II:
\begin{enumerate}
	\item Truy xuất dựa vào một biến bit.
	\item Truy xuất bằng cấu trúc truy xuất trực tiếp một bit của thanh ghi. Ví dụ: ACC.5
\end{enumerate}

Tiêu chí III:
\begin{enumerate}
	\item Không vi phạm nguyên tắc sử dụng. Nghĩa là: mỗi một biến bit chỉ được sử dụng khi thanh ghi ACC đang mang giá trị vùng nhớ có địa chỉ quy định bởi một biến byte duy nhất. Và không có hai biến bit nào cùng vị trí được sử dụng khi thanh ghi ACC đang mang giá trị của một biến byte nào đó.
	\item Vi phạm nguyên tắc sử dụng. Một biến bit được sử dụng ở nhiều chỗ, trong các chỗ đó thanh ghi ACC mang giá trị của những biết byte khác nhau.
	\item Vi phạm nguyên tắc sử dụng. Một biến bit được sử dụng ở một vị trí, tại vị trí đó thanh ghi ACC có thể mang giá trị của nhiều vùng nhớ khác nhau, không thể xác định trước khi thực thi chương trình.
	\item Vi phạm nguyên tắc sử dụng. Một biến byte được sử dụng ở nhiều vị trí, sau câu lệnh gán biến byte, có 2 biến bit cùng một vị trí được sử dụng.
\end{enumerate}

Dựa vào các tiêu chí và trường hợp trên, ta sẽ có tổng cộng 5x2x4 = 40 loại testcase. Ngoài ra, sẽ có một testcase phức tạp được lấy từ một đoạn chương trình thực của doanh nghiệp được đưa vào kiểm thử, nhằm đảm bảo tính thực tế của luận văn này.

%bảng phân bổ số lượng testcase

\section{Phương thức kiểm thử}
Vì số lượng testcase không quá lớn, và output ra của trình dịch ngược có khá nhiều thông tin khác ngoài phạm vi luận văn, nên ta sẽ dùng cách kiểm tra bằng tay. Hai bộ source code sẽ được đưa vào vòng lặp, chạy từ testcase 1 đến testcase 50. Output ở console sẽ được lưu vào file có định dạng [số thứ tự testcase]console.txt và đoạn mã đầu ra (nếu có) sẽ được lưu vào file có định dạng [số thứ tự testcase]code.txt. Sau đó người viết sẽ trực tiếp kiểm tra hai file này để xác định kết quả có đúng như mong muốn hay không. 

%ví dụ đoạn mã đầu vào và 2 file output

\section{Kết quả chạy thử}

Kết quả chạy thử của 2 phương pháp được thể hiện ở bảng dưới.

%bảng kết quả chạy thử

Như vậy, có thể thấy giải pháp đầu tiên ra kết quả không chính xác rất nhiều, còn giải pháp Suy luận kiểu thì ra được kết quả chấp nhận được. Điều này đã được dự báo trước vì giải pháp Kiểm tra kiểu còn nhiều hạn chế.

	\chapter{Kết luận}

Chương này sẽ tổng kết lại các kết quả đã đạt được của luận văn và đưa ra hướng phát triển trong tương lai.

\section{Kết quả đạt được, khó khăn, điểm hạn chế}

Nhìn chung, luận văn đã hoàn thành mục tiêu đề ra ban đầu, giải quyết được bài toán về kiểu dữ liệu bit và câu lệnh xử lý bit trong mã assembly của 8051. Ngoài ra, luận văn đã chứng minh được tính thực tiễn của đề tài, khả năng áp dụng vào thực tế của các doanh nghiệp có nhu cầu dịch ngược. Cuối cùng, luận văn cũng đưa ra một phương pháp lập testcase và kiểm thử khoa học, đảm bảo đưa ra được các trường hợp có thể xảy ra ngoài thực tế và đặc biệt có một testcase là một đoạn mã thật của doanh nghiệp. Kết quả kiểm thử trên bộ testcase dành cho phương pháp Suy luận kiểu là chấp nhận được.

\section{Hướng phát triển trong tương lai}

Các hướng phát triển trong tương lai của trình dịch ngược gồm có:

\begin{itemize}
	\item Tiếp tục mở rộng khả năng dịch ngược cho nhiều máy khác nhau
	\item Phân tích và sửa lỗi sai của giải thuật phân tích dòng dữ liệu
	\item Cải tiến chức năng nhận dạng kiểu của Boomerang
\end{itemize}

	
	\newpage
	\nocite{*}
	\printbibliography

\end{document}